\subsection{if/else 条件の使用}


プログラミングの重要な部分は、データに基づいて意思決定を行うことです。 このユニットでは、条件をテストしてプログラムのアクションを制御する方法について説明します。

\texttt{if} および \texttt{else} キーワードを使用して、コード内に "条件分岐" を作成できます。 多くのプログラミング言語によってこの機能が提供されており、同様の構文が使用されます。

\texttt{if} および \texttt{else} キーワードは、値をテストし、テスト結果に基づいてアクションを実行するために式と共に使用されます。 すべての条件式の結果は、ブール値の true または false になります。

\subsubsection{If/else 条件を定義する}

2 つの数値が等しいかどうかをテストし、テスト結果に基づいてメッセージを出力する例を次に示します。



\begin{lstlisting}[numbers=none]
if 1 == 2 {
    println!("True, the numbers are equal."); // 
} else {
    println!("False, the numbers are not equal.");
}
\end{lstlisting}

前の例では、 \texttt{if} の条件は式 \texttt{1 == 2} で、これは false の値を持つブール型に評価されます。

他のほとんどの言語とは異なり、Rust の \texttt{if} ブロックは式としても機能できます。 条件分岐内のすべての実行ブロックでは、コードをコンパイルするために同じ型を返す必要があります。



\begin{lstlisting}[numbers=none]
let formal = true;
let greeting = if formal { // if used here as an expression
    "Good day to you."     // return a String
} else {
    "Hey!"                 // return a String
};
println!("{}", greeting)   // prints "Good day to you."
\end{lstlisting}

この例では、\texttt{if formal} 式の結果に基づいて \texttt{greeting} 変数に値を代入します。 式 \texttt{if formal} が true の場合、 \texttt{greeting} 値は "Good day to you" という文字列に設定されます。 式が false の場合、 \texttt{greeting} 値は文字列 "Hey!" に設定されます。 \texttt{formal} 変数を true に初期化したので、式 \texttt{if formal} の結果も true になることがわかります。

\subsubsection{複数のテスト条件を組み合わせる}

\texttt{if} と \texttt{else} を組み合わせて、\texttt{else if} 式を作成することができます。 最初の \texttt{if} 条件の後、および省略可能な最後の \texttt{else} 条件の前に、複数の \texttt{else if} 条件を使用できます。

条件式が \texttt{true} に評価された場合、対応するアクションのブロックが実行されます。 次の \texttt{else if} または \texttt{else} ブロックはスキップされます。 条件式が \texttt{false} に評価された場合、対応するアクションのブロックがスキップされます。 次の \texttt{else if} 条件が評価されます。 すべての \texttt{if} 条件と \texttt{else if} 条件が \texttt{false} に評価された場合、 \texttt{else} ブロックが実行されます。

この例では、数値が許容範囲内であるかどうかを確認します。 数値が 0 未満、0 に等しい、または 512 より大きい場合に、特定の処理を実行します。 ブール型変数 \texttt{out\_of\_range} を宣言しますが、プログラムが条件テスト式に入るまで変数の値は設定しません。


\begin{lstlisting}[numbers=none]
let num = 500 // num変数は、プログラムのある時点で設定することができる
let out_of_range: bool;
if num < 0 {
    out_of_range = true;
} else if num == 0 {
    out_of_range = true;
} else if num > 512 {
    out_of_range = true;
} else {
    out_of_range = false;
}
\end{lstlisting}

次の演習では、\texttt{if} および \texttt{else} 条件式を使用します。

