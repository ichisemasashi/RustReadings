\subsection{モジュールを別のファイルに分割する}

モジュールの内容が大きくなりすぎると、コード ナビゲーションがより困難になります。 モジュールの内容を別のファイルに移動することを検討してください。

前の例のコードを \texttt{src/authentication.rs} という名前の独自のファイルに移動して、クレートのルート ファイルを変更しましょう。

\begin{lstlisting}[numbers=none, caption=src/main.rs]
mod authentication;

fn main() {
    let mut user = authentication::User::new("jeremy", "super-secret");

    println!("The username is: {}", user.get_username());
    user.set_password("even-more-secret");
}
\end{lstlisting}

\begin{lstlisting}[numbers=none, caption=src/authentication.rs]
pub struct User {
    username: String,
    password_hash: u64,
}

impl User {
    pub fn new(username: &str, password: &str) -> User {
        User {
            username: username.to_string(),
            password_hash: hash_password(&password.to_owned()),
        }
    }

    pub fn get_username(&self) -> &String {
        &self.username
    }

    pub fn set_password(&mut self, new_password: &str) {
        self.password_hash = hash_password(&new_password.to_owned())
    }
}

fn hash_password<T: Hash>(t: &T) -> u64 {/* ... */}
\end{lstlisting}

\texttt{mod authentication} の後には、コード ブロックの代わりにセミコロンを配置します。 ファイルのサイズが大きくなると、この手法を使用することで、モジュールを新しいファイルに自動的に移動できます。 コンパイラによって、モジュールと同じ名前の別のファイルからモジュールの内容が読み込まれます。

コンテンツが読み込まれても、モジュール ツリーは同じままです。 コードは、定義が異なるファイルに存在する場合でも動作し、変更する必要はありません。








