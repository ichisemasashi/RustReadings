\subsection{演習: ハッシュ マップを使用して注文を追跡する}

この演習では、ハッシュ マップを使用するために自動車工場プログラムを変更します。

ハッシュ マップ キーと値のペアを使用して、自動車の注文に関する詳細を追跡し、出力を表示します。 ここでも、課題は、コンパイルして実行できるようにサンプル コードを完成させることです。

この演習用のサンプル コードを操作するには、次の 2 つの方法があります。

\begin{itemize}
\item コードをコピーし、ローカルの開発環境で編集します。
\item 準備済みの Rust Playground 内でコードを開きます。
\end{itemize}

\begin{itembox}[l]{注意}
サンプル コードで、\texttt{todo!} マクロを探します。 このマクロは、完了または更新する必要があるコードを示しています。
\end{itembox}

\subsubsection{現在のプログラムを読み込む}

最初の手順では、既存のプログラム コードを取得します。

\begin{enumerate}

\item 編集する既存のプログラム コードを開きます。 このコードには、データ型の宣言と、\texttt{car\_quality}、\texttt{car\_factory}、および \texttt{main} 関数の定義が含まれています。

次のコードをコピーし、ローカルの開発環境で編集します。
または、この準備済みの Rust Playground 内でコードを開きます。


\begin{lstlisting}[numbers=none]
#[derive(PartialEq, Debug)]
struct Car { color: String, motor: Transmission,
             roof: bool, age: (Age, u32) }

#[derive(PartialEq, Debug)]
enum Transmission { Manual, SemiAuto, Automatic }

#[derive(PartialEq, Debug)]
enum Age { New, Used }

// 入力引数の値をテストして車の品質を取得する
// - miles (u32)
// 車齢("新車 "または "中古車")と走行距離のタプルを返す。
fn car_quality (miles: u32) -> (Age, u32) {

    // 走行距離が伸びていないか
    // 中古車のための初期のタプルを返す
    if miles > 0 {
        return (Age::Used, miles);
    }

    // 新車のタプルを返す。returnキーワードやセミコロンは不要
    (Age::New, miles)
}

// 入力引数で "Car "をビルドする
fn car_factory(order: i32, miles: u32) -> Car {
    let colors = ["Blue", "Green", "Red", "Silver"];

    // パニックを防止する。colors 配列のカラーインデックスをチェックし、
    // 必要に応じてリセットする。
    // Valid color = 1, 2, 3, or 4
    // If color > 4, reduce color to valid index
    let mut color = order as usize;
    if color > 4 {        
        // color = 5 --> index 1, 6 --> 2, 7 --> 3, 8 --> 4
        color = color - 4;
    }

    // モータータイプ、ルーフタイプの注文にバラエティーを持たせる。
    let mut motor = Transmission::Manual;
    let mut roof = true;
    if order % 3 == 0 {          // 3, 6, 9
        motor = Transmission::Automatic;
    } else if order % 2 == 0 {   // 2, 4, 8, 10
        motor = Transmission::SemiAuto;
        roof = false;
    }                            // 1, 5, 7, 11

    // Return requested "Car"
    Car {
        color: String::from(colors[(color-1) as usize]),
        motor: motor,
        roof: roof,
        age: car_quality(miles)
    }
}

fn main() {
    // カウンタ変数の初期化
    let mut order = 1;
    // 自動車をミュータブルな「Car」構造体として宣言する。
    let mut car: Car;

    // 6台の車を注文、リクエストごとに "order "をインクリメント
    // Car order #1: Used, Hard top
    car = car_factory(order, 1000);
    println!("{}: {:?}, Hard top = {}, {:?}, {}, {} miles",
     order, car.age.0, car.roof, car.motor, car.color, car.age.1);

    // Car order #2: Used, Convertible
    order = order + 1;
    car = car_factory(order, 2000);
    println!("{}: {:?}, Hard top = {}, {:?}, {}, {} miles",
     order, car.age.0, car.roof, car.motor, car.color, car.age.1);    

    // Car order #3: New, Hard top
    order = order + 1;
    car = car_factory(order, 0);
    println!("{}: {:?}, Hard top = {}, {:?}, {}, {} miles",
     order, car.age.0, car.roof, car.motor, car.color, car.age.1);

    // Car order #4: New, Convertible
    order = order + 1;
    car = car_factory(order, 0);
    println!("{}: {:?}, Hard top = {}, {:?}, {}, {} miles",
     order, car.age.0, car.roof, car.motor, car.color, car.age.1);

    // Car order #5: Used, Hard top
    order = order + 1;
    car = car_factory(order, 3000);
    println!("{}: {:?}, Hard top = {}, {:?}, {}, {} miles",
     order, car.age.0, car.roof, car.motor, car.color, car.age.1);

    // Car order #6: Used, Hard top
    order = order + 1;
    car = car_factory(order, 4000);
    println!("{}: {:?}, Hard top = {}, {:?}, {}, {} miles",
     order, car.age.0, car.roof, car.motor, car.color, car.age.1);
}
\end{lstlisting}

\item プログラムをビルドします。 次のセクションに進む前に、コードがコンパイルされ、実行されることを確認してください。
\end{enumerate}


次の出力が表示されます。


\begin{lstlisting}[numbers=none]
1: Used, Hard top = true, Manual, Blue, 1000 miles
2: Used, Hard top = false, SemiAuto, Green, 2000 miles
3: New, Hard top = true, Automatic, Red, 0 miles
4: New, Hard top = false, SemiAuto, Silver, 0 miles
5: Used, Hard top = true, Manual, Blue, 3000 miles
6: Used, Hard top = true, Automatic, Green, 4000 miles
\end{lstlisting}

\subsubsection{注文の詳細を追跡するハッシュ マップを追加する}

現在のプログラムでは、各自動車の注文が満たされ、各注文の完了後に概要が出力されます。 \texttt{car\_factory} 関数を呼び出すたびに、注文の詳細を含む \texttt{Car} 構造体が返されて注文が満たされます。 結果は \texttt{car} 変数に格納されます。

お気付きかもしれませんが、プログラムにはいくつかの重要な機能がありません。 すべての注文を追跡しているわけではありません。 \texttt{car} 変数には、現在の注文の詳細のみが保持されます。 \texttt{car\_factory} 関数からの結果で \texttt{car} 変数が更新されるたびに、前の注文の詳細が上書きされます。

ファイリング システムの場合のようにすべての注文を追跡するには、プログラムを更新する必要があります。 このために、\texttt{<K, V>} ペアでハッシュ マップを定義します。 ハッシュ マップ キーは自動車の注文番号に対応します。 ハッシュ マップ値は、 \texttt{Car} 構造体で定義されているそれぞれの注文の詳細となります。

\begin{enumerate}
\item ハッシュ マップを定義するには、 \texttt{main} 関数の先頭の左中かっこ \texttt{\{} の直後に次のコードを追加します。


\begin{lstlisting}[numbers=none]
    // 車の注文のためのハッシュマップの初期化
    // - Key: Car order number, i32
    // - Value: Car order details, Car struct
    use std::collections::HashMap;
    let mut orders: HashMap<i32, Car> = HashMap;
\end{lstlisting}

\item \texttt{orders} ハッシュ マップを作成するステートメントの構文の問題を修正します。

\begin{itembox}[l]{ヒント}
ハッシュ マップを最初から作成しているので、おそらく \texttt{new()} メソッドを使用します。
\end{itembox}

\item プログラムをビルドします。 次のセクションに進む前に、コードがコンパイルされることを確認してください。 コンパイラからの警告メッセージは無視してかまいません。

\end{enumerate}

\subsubsection{ハッシュ マップに値を追加する}

次の手順では、ハッシュ マップに自動車の満たされた各注文を追加します。

\texttt{main} 関数では、自動車の各注文に対して \texttt{car\_factory} 関数を呼び出します。 注文が満たされた後、\texttt{println!} マクロを呼び出して、 \texttt{car} 変数に格納されている注文の詳細を表示します。

\begin{lstlisting}[numbers=none]
    // Car order #1: Used, Hard top
    car = car_factory(order, 1000);
    println!("{}: {}, Hard top = {}, {:?}, {}, {} miles",
     order, car.age.0, car.roof, car.motor, car.color, car.age.1);

    ...

    // Car order #6: Used, Hard top
    order = order + 1;
    car = car_factory(order, 4000);
    println!("{}: {}, Hard top = {}, {:?}, {}, {} miles",
     order, car.age.0, car.roof, car.motor, car.color, car.age.1);
\end{lstlisting}

新しいハッシュ マップを操作するために、これらのコード ステートメントを修正します。

\begin{itemize}

\item \texttt{car\_factory} 関数の呼び出しを保持する。 返された各 \texttt{Car} 構造体は、ハッシュ マップに <K, V> ペアの一部として格納されます。

\item \texttt{println!} マクロの呼び出しを更新して、ハッシュ マップに格納されている注文の詳細を表示する。

\end{itemize}

\begin{enumerate}

\item \texttt{main} 関数で、 \texttt{car\_factory} 関数の呼び出しと、 \texttt{println!} マクロの付随する呼び出しを見つけます。

\begin{lstlisting}[numbers=none]
    // Car order #1: Used, Hard top
    car = car_factory(order, 1000);
    println!("{}: {}, Hard top = {}, {:?}, {}, {} miles",
     order, car.age.0, car.roof, car.motor, car.color, car.age.1);

    ...

    // Car order #6: Used, Hard top
    order = order + 1;
    car = car_factory(order, 4000);
    println!("{}: {}, Hard top = {}, {:?}, {}, {} miles",
     order, car.age.0, car.roof, car.motor, car.color, car.age.1);
\end{lstlisting}

\item すべての自動車注文のステートメントの完全なセットを、次の修正後のコードに置き換えます。

\begin{lstlisting}[numbers=none]
    // Car order #1: Used, Hard top
    car = car_factory(order, 1000);
    orders(order, car);
    println!("Car order {}: {:?}", order, orders.get(&order));

    // Car order #2: Used, Convertible
    order = order + 1;
    car = car_factory(order, 2000);
    orders(order, car);
    println!("Car order {}: {:?}", order, orders.get(&order));

    // Car order #3: New, Hard top
    order = order + 1;
    car = car_factory(order, 0);
    orders(order, car);
    println!("Car order {}: {:?}", order, orders.get(&order));

    // Car order #4: New, Convertible
    order = order + 1;
    car = car_factory(order, 0);
    orders(order, car);
    println!("Car order {}: {:?}", order, orders.get(&order));

    // Car order #5: Used, Hard top
    order = order + 1;
    car = car_factory(order, 3000);
    orders(order, car);
    println!("Car order {}: {:?}", order, orders.get(&order));

    // Car order #6: Used, Hard top
    order = order + 1;
    car = car_factory(order, 4000);
    orders(order, car);
    println!("Car order {}: {:?}", order, orders.get(&order));
\end{lstlisting}

\item ここでプログラムをビルドしようとすると、コンパイル エラーが表示されます。 <K, V> ペアを \texttt{orders} ハッシュ マップに追加するステートメントには、構文の問題があります。 その問題がわかりますか。 それでは、注文をハッシュ マップに追加する各ステートメントの問題を修正しましょう。

\begin{itembox}[l]{ヒント}
\texttt{orders} ハッシュ マップに値を直接割り当てることはできません。 挿入を行うには、メソッドを使用する必要があります。
\end{itembox}


\end{enumerate}

\subsubsection{プログラムの実行}

プログラムが正常にビルドされた後、次の出力が表示されるはずです。

\begin{lstlisting}[numbers=none]
Car order 1: Some(Car { color: "Blue", motor: Manual,
                        roof: true, age: ("Used", 1000) })
Car order 2: Some(Car { color: "Green", motor: SemiAuto,
                        roof: false, age: ("Used", 2000) })
Car order 3: Some(Car { color: "Red", motor: Automatic,
                        roof: true, age: ("New", 0) })
Car order 4: Some(Car { color: "Silver", motor: SemiAuto,
                        roof: false, age: ("New", 0) })
Car order 5: Some(Car { color: "Blue", motor: Manual,
                        roof: true, age: ("Used", 3000) })
Car order 6: Some(Car { color: "Green", motor: Automatic,
                        roof: true, age: ("Used", 4000) })
\end{lstlisting}

変更後のコードの出力が異なることに注目してください。 \texttt{println!} マクロでは、各値と対応するフィールド名を表示することによって、 \texttt{Car} 構造体の内容を表示します。

次の演習では、ループ式を使用してコード内の冗長性を減らします。

\subsubsection{解決策}

この Rust Playground 内で、プログラム出力をこの演習のソリューションと比較できます。

