\subsection{演習: ハッシュ マップを使用して注文を追跡する}

この演習では、ハッシュ マップを使用するために自動車工場プログラムを変更します。

ハッシュ マップ キーと値のペアを使用して、自動車の注文に関する詳細を追跡し、出力を表示します。 ここでも、課題は、コンパイルして実行できるようにサンプル コードを完成させることです。

この演習用のサンプル コードを操作するには、次の 2 つの方法があります。

\begin{itemize}
\item コードをコピーし、ローカルの開発環境で編集します。
\item 準備済みの Rust Playground 内でコードを開きます。
\end{itemize}

\begin{itembox}[l]{注意}
サンプル コードで、\texttt{todo!} マクロを探します。 このマクロは、完了または更新する必要があるコードを示しています。
\end{itembox}

\subsubsection{現在のプログラムを読み込む}

最初の手順では、既存のプログラム コードを取得します。

\begin{enumerate}

\item 編集する既存のプログラム コードを開きます。 このコードには、データ型の宣言と、\texttt{car\_quality}、\texttt{car\_factory}、および \texttt{main} 関数の定義が含まれています。

次のコードをコピーし、ローカルの開発環境で編集します。
または、この準備済みの Rust Playground 内でコードを開きます。


\begin{lstlisting}[numbers=none]
#[derive(PartialEq, Debug)]
struct Car { color: String, motor: Transmission,
             roof: bool, age: (Age, u32) }

#[derive(PartialEq, Debug)]
enum Transmission { Manual, SemiAuto, Automatic }

#[derive(PartialEq, Debug)]
enum Age { New, Used }

// 入力引数の値をテストして車の品質を取得する
// - miles (u32)
// 車齢("新車 "または "中古車")と走行距離のタプルを返す。
fn car_quality (miles: u32) -> (Age, u32) {

    // 走行距離が伸びていないか
    // 中古車のための初期のタプルを返す
    if miles > 0 {
        return (Age::Used, miles);
    }

    // 新車のタプルを返す。returnキーワードやセミコロンは不要
    (Age::New, miles)
}

// 入力引数で "Car "をビルドする
fn car_factory(order: i32, miles: u32) -> Car {
    let colors = ["Blue", "Green", "Red", "Silver"];

    // パニックを防止する。colors 配列のカラーインデックスをチェックし、
    // 必要に応じてリセットする。
    // Valid color = 1, 2, 3, or 4
    // If color > 4, reduce color to valid index
    let mut color = order as usize;
    if color > 4 {        
        // color = 5 --> index 1, 6 --> 2, 7 --> 3, 8 --> 4
        color = color - 4;
    }

    // モータータイプ、ルーフタイプの注文にバラエティーを持たせる。
    let mut motor = Transmission::Manual;
    let mut roof = true;
    if order % 3 == 0 {          // 3, 6, 9
        motor = Transmission::Automatic;
    } else if order % 2 == 0 {   // 2, 4, 8, 10
        motor = Transmission::SemiAuto;
        roof = false;
    }                            // 1, 5, 7, 11

    // Return requested "Car"
    Car {
        color: String::from(colors[(color-1) as usize]),
        motor: motor,
        roof: roof,
        age: car_quality(miles)
    }
}

fn main() {
    // カウンタ変数の初期化
    let mut order = 1;
    // 自動車をミュータブルな「Car」構造体として宣言する。
    let mut car: Car;

    // 6台の車を注文、リクエストごとに "order "をインクリメント
    // Car order #1: Used, Hard top
    car = car_factory(order, 1000);
    println!("{}: {:?}, Hard top = {}, {:?}, {}, {} miles",
     order, car.age.0, car.roof, car.motor, car.color, car.age.1);

    // Car order #2: Used, Convertible
    order = order + 1;
    car = car_factory(order, 2000);
    println!("{}: {:?}, Hard top = {}, {:?}, {}, {} miles",
     order, car.age.0, car.roof, car.motor, car.color, car.age.1);    

    // Car order #3: New, Hard top
    order = order + 1;
    car = car_factory(order, 0);
    println!("{}: {:?}, Hard top = {}, {:?}, {}, {} miles",
     order, car.age.0, car.roof, car.motor, car.color, car.age.1);

    // Car order #4: New, Convertible
    order = order + 1;
    car = car_factory(order, 0);
    println!("{}: {:?}, Hard top = {}, {:?}, {}, {} miles",
     order, car.age.0, car.roof, car.motor, car.color, car.age.1);

    // Car order #5: Used, Hard top
    order = order + 1;
    car = car_factory(order, 3000);
    println!("{}: {:?}, Hard top = {}, {:?}, {}, {} miles",
     order, car.age.0, car.roof, car.motor, car.color, car.age.1);

    // Car order #6: Used, Hard top
    order = order + 1;
    car = car_factory(order, 4000);
    println!("{}: {:?}, Hard top = {}, {:?}, {}, {} miles",
     order, car.age.0, car.roof, car.motor, car.color, car.age.1);
}
\end{lstlisting}

\item プログラムをビルドします。 次のセクションに進む前に、コードがコンパイルされ、実行されることを確認してください。
\end{enumerate}


次の出力が表示されます。


\begin{lstlisting}[numbers=none]
1: Used, Hard top = true, Manual, Blue, 1000 miles
2: Used, Hard top = false, SemiAuto, Green, 2000 miles
3: New, Hard top = true, Automatic, Red, 0 miles
4: New, Hard top = false, SemiAuto, Silver, 0 miles
5: Used, Hard top = true, Manual, Blue, 3000 miles
6: Used, Hard top = true, Automatic, Green, 4000 miles
\end{lstlisting}

\subsubsection{注文の詳細を追跡するハッシュ マップを追加する}

