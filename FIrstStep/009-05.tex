\subsection{サードパーティーのクレートをプロジェクトに追加する}

このモジュールでは、サードパーティーのクレートをプロジェクトに追加する方法について学習します。 Rust の標準ライブラリには "正規表現" のモジュールがないため、Crates.io で利用できる \texttt{regex} のクレートを追加してみましょう。 この Web サイトは、パッケージを検出してダウンロードするための場所として機能する、Rust コミュニティの中央パッケージ レジストリです。

プロジェクトに依存するクレートを追加するときは常に、最も手間のかかる部分に Cargo を活用することができます。 crates.io でホストされているライブラリに依存させるには、 \texttt{Cargo.toml} ファイルに追加します。

\begin{lstlisting}[numbers=none]
[dependencies]
regex = "1.4.2"
\end{lstlisting}

Cargo.toml に \texttt{[dependencies]} セクションがまだない場合は、そのセクションを追加します。 次に、使用するクレートの名前とバージョンを一覧表示します。

次のステップは、コマンド \texttt{cargo build} を実行することです。 Cargo は、新しい依存関係とそのすべての依存関係を取得し、それらすべてをコンパイルします。

\begin{lstlisting}[numbers=none]
    $ cargo build
        Updating crates.io index
      Downloaded regex v1.4.2
      Downloaded thread_local v1.0.1
      Downloaded regex-syntax v0.6.21
      Downloaded lazy_static v1.4.0
      Downloaded aho-corasick v0.7.15
      Downloaded memchr v2.3.4
      Downloaded 6 crates (689.7 KB) in 4.58s
       Compiling memchr v2.3.4
       Compiling lazy_static v1.4.0
       Compiling regex-syntax v0.6.21
       Compiling thread_local v1.0.1
       Compiling aho-corasick v0.7.15
       Compiling regex v1.4.2
       Compiling my-project v0.1.0 (/home/user/code/my-project)
        Finished dev [unoptimized + debuginfo] target(s) in 35.13s
\end{lstlisting}

これで、\texttt{main.rs} で \texttt{regex} ライブラリを使用できるようになりました。

\begin{lstlisting}[numbers=none]
use regex::Regex;

fn main() {
    let re = Regex::new(r"^\d{4}-\d{2}-\d{2}$").unwrap();
    println!("Did our date match? {}", re.is_match("2014-01-01"));
}
\end{lstlisting}

実行すると、次のように表示されます。

\begin{lstlisting}[numbers=none]
    $ cargo run
       Running `target/hello_world`
    Did our date match? true
\end{lstlisting}

また、Rust Playground でサードパーティーのクレートを使用することもできます。














