\subsection{知識チェック}

次の質問に答えて、学習した内容を確認してください。

\begin{enumerate}

\item 次のうち、試しても Rust プログラムがパニックに "ならない" ものはどれですか?

\begin{itemize}
\item \texttt{vector[index]} という表記を使用して、ベクターの範囲外のインデックスにアクセスする。
\item \texttt{vector.get(index)} という表記を使用して、ベクターの範囲外のインデックスにアクセスする。
\item \texttt{array[index]} という表記を使用して、配列の範囲外のインデックスにアクセスする。
\item \texttt{HashMap[key]} という表記を使用して、ハッシュ マップの存在しないキーにアクセスする。
\end{itemize}

\item Rust で、特定の型 \texttt{T} に値が存在しない可能性があることは、どのようにして表すことができますか?

\begin{itemize}
\item \texttt{Option<T>} 型です。
\item \texttt{Result<T, bool>} 型です。
\item \texttt{bool} 型の \texttt{false} 値。
\item 空のタプル: \texttt{()}。
\end{itemize}

\item 特定の型 \texttt{T} の値を取得しているときの、入出力 (I/O) エラーの可能性は、どのようにして表すことができますか?

\begin{itemize}
\item \texttt{Option<T>} 型です。
\item \texttt{Result<T, io::Error>} 型です。
\item 空の \texttt{Vec<T>}。
\item 空のタプル: \texttt{()}。
\end{itemize}

\end{enumerate}

