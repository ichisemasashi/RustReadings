\subsection{Rust とは}

Rust は、効率的で安全なソフトウェアの開発に使用できるオープンソースのシステム プログラミング言語です。 Rust を使用すると、メモリを管理し、その他の低レベルの詳細を制御できます。 ただし、イテレーションやインターフェイスなど、高レベルの概念も活用できます。 こうした機能が Rust と、C や C++ のような低レベル言語との違いになります。

さらに、Rust には、幅広いアプリケーションで使用することを可能にする次のような利点があります。

\begin{itemize}
\item \textbf{タイプ セーフ}: コンパイラによって、型が間違っている変数に操作が適用されないことが保証されます。
\item \textbf{メモリ セーフ}: Rust ポインター ("参照" と呼ばれます) によって、常に有効なメモリが参照されます。
\item \textbf{データ競合なし}: プログラムの複数の部分が同時に同じ値を変化させることができないようにする Rust のボロー チェッカーによって、スレッドセーフが保証されます。
\item \textbf{ゼロコスト抽象化}: Rust では、イテレーションやインターフェイス、関数型プログラミングなどの高度な概念を、パフォーマンス コストをほとんど、またはまったくかけずに利用できます。 抽象化は、基になるコードを手動で記述した場合と同様に実行されます。
\item \textbf{最小限のランタイム}: Rust のランタイムはごく小さく、省略可能です。 この言語には、メモリを効率的に管理するためのガベージ コレクターもありません。 このように、Rust は C や C++ などの言語とよく似ています。
\item \textbf{ターゲットはベアメタル}: Rust では、組み込みの "ベアメタル" プログラミングをターゲットにすることができます。これにより、オペレーティング システムのカーネルまたはデバイス ドライバーを記述するのに適しています。
\end{itemize}

2021 年の Stack Overflow による開発者アンケートによれば、Rust は、数年連続で最も好まれている言語になっています。 開発者は Rust を使用してプログラミングを楽しんでいます。 スタートアップ企業から大企業まで、さまざまな種類の組織が独自の用途に Rust を使用しています。 ツールの構築から Web アプリの作成、サーバーでの作業、組み込みシステムの作成に至るまで、その可能性は無限です。