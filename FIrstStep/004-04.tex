\subsection{演習: 複合型を処理する}

この演習では、複合データ型を使用して、自動車工場プログラムを拡張します。

タプルを使用して、関連するがデータ型が異なる 2 つの値を使用して自動車の品質を追跡します。 このタプルをこの関数の呼び出し元に返す \texttt{car\_quality} という名前の関数を作成します。 \texttt{main} 関数で、 \texttt{car\_factory} 関数を呼び出して各自動車注文を作成します。

課題は、コンパイルして実できるようにサンプル コードを完成させることです。

この演習用のサンプル コードを操作するには、次の 2 つの方法があります。

\begin{itemize}
\item コードをコピーし、ローカルの開発環境で編集します。
\item 準備済みの Rust Playground 内でコードを開きます。
\end{itemize}

\begin{itembox}[l]{注意}
サンプル コードで、\texttt{todo!} マクロを探します。 このマクロは、完了または更新する必要があるコードを示しています。
\end{itembox}

\subsubsection{タプル フィールドを持つように Car 構造体を更新する}

最初のタスクは、\texttt{Car} 構造体の定義の変更です。 \texttt{mileage} フィールドを \texttt{age} という名前のタプル フィールドに移動します。 走行距離値と共に、\texttt{age} タプルには、自動車が "新車" か "中古" かを識別するための別のフィールドが必要です。

\begin{enumerate}
\item サンプル コードの最初のブロックを開きます。

次のコードをコピーしてローカルの開発環境で編集するか、この用意されている Rust プレイグラウンドでコードを開きます。


\begin{lstlisting}[numbers=none]
#[derive(PartialEq, Debug)]
// Car構造体を宣言し、4つの名前付きフィールドで車両を記述する。
struct Car {
    color: String,
    motor: Transmission,
    roof: bool,
    mileage: u32, // todo!("Move `mileage: u32` field into `age` field\\
                    - a tuple with two fields: an `Age` enum, u32");
}

#[derive(PartialEq, Debug)]
// Carトランスミッションの種類を表すenumを宣言する
enum Transmission { Manual, SemiAuto, Automatic }
\end{lstlisting}

\item 自動車の品質を記述するために、"新車" と "中古" の値を持つ \texttt{Age} という名前の列挙型を追加します。

\item \texttt{Car} 構造体の宣言を修正します。

\begin{enumerate}
\item \texttt{mileage: u32} フィールドを \texttt{age} という名前のタプル フィールドに置き換えます。
\item \texttt{Age} 列挙値と自動車走行距離の 2 つのフィールドを持つ \texttt{age} タプルを定義します。
\end{enumerate}

\item プログラムをビルドします。 次のセクションに進む前に、コードがコンパイルされることを確認してください。

コードではまだ出力は表示されませんが、エラーなしでコンパイルされる必要があります。 コンパイラからの "警告" メッセージは無視してかまいません。 警告は、列挙型と構造体の定義を宣言したものの、これらをまだ使用していないことが原因で生成されます。
\end{enumerate}


\subsubsection{car\_quality 関数を作成する}

次に、 \texttt{car\_quality} という名前の新しい関数のコードを追加します。 この関数は、自動車走行距離を入力引数として受け取ります。 走行距離と自動車年数を保持するタプルを作成します。 関数から呼び出し元にタプルが返されます。

\begin{enumerate}

\item 既存のコードに次のコード ブロックを追加します。 新しいコードは、ファイルの先頭または末尾に追加できます。

\begin{lstlisting}[numbers=none]
// 入力引数の値をテストして車の品質を取得する
// - miles (u32)
// 車の品質について、Age("New "または "Used")と走行距離のタプルを作成する。
// 矢印 `->` 構文でタプルを返す。
fn car_quality (miles: u32) -> (Age, u32) {

    // 戻り値タプル値の宣言と初期化
    // 新車の場合、マイルを0に設定する
    let quality: (Age, u32) = todo!("Set the `Age` value to \"New\",\\
                 set the mileage using the `miles` input argument");

    // 完成したタプルを呼び出し元に返す
    todo!("Return the tuple");
}
\end{lstlisting}

\item "新車" の \texttt{quality} タプル値を設定するコードを完成させます。

\item 関数の末尾にある \texttt{return} ステートメントを更新して、完成したタプルを呼び出し元に返します。

\item プログラムをビルドします。 次のセクションに進む前に、コードがコンパイルされることを確認してください。 コードではまだ出力は表示されませんが、エラーなしでコンパイルされる必要があります。


\end{enumerate}


\subsubsection{car\_factory 関数を更新する}

次の手順では、 \texttt{car\_factory} 関数を更新します。 \texttt{car\_quality} 関数から返されるタプルをサポートする必要があります。 \texttt{Car} 構造体の定義を更新したので、データを正しく処理するように関数本体を調整する必要があります。

\begin{enumerate}

\item 既存のコードに次のコード ブロックを追加します。 新しいコードは、ファイルの先頭または末尾に追加できます。

\begin{lstlisting}[numbers=none]
// 4つの入力引数の値を使って新しい "Car "を作る
// - color (String)
// - motor (Transmission enum)
// - roof (boolean, true if the car has a hard top roof)
// - miles (u32)
// car_quality(miles)関数を呼び出して車齢を取得する
// 矢印 `->` 構文を持つ "Car" 構造体のインスタンスを返す。
fn car_factory(color: String, motor: Transmission, roof: bool, miles: u32) -> Car {
    // 要求に応じて新しい "Car "インスタンスを作成する
    // - 最初の3つのフィールドを入力引数の値にバインドする
    // - "age "フィールドが "miles "入力引数で "car_quality "関数を呼び出す。
    Car {
        color: color,
        motor: motor,
        roof: roof,
        mileage: miles, // todo!("Replace `mileage: miles` with \\
        `age` tuple field, call `car_quality()` with `miles` as \\
        input argument");
    }
}
\end{lstlisting}

\item \texttt{car} 変数の初期化を修正します。

\begin{enumerate}
\item \texttt{mileage: miles} フィールドを \texttt{age} という名前のタプル フィールドに置き換えます。
\item \texttt{age} フィールドは、 \texttt{car\_quality} 関数を \texttt{miles} 入力引数を使用して呼び出す必要があります。
\end{enumerate}

\item プログラムをビルドします。 次のセクションに進む前に、コードがコンパイルされることを確認してください。

\end{enumerate}

\subsubsection{main 関数の変数を設定する}

これで、\texttt{main} 関数の操作を開始する準備ができました。 最初の手順で、プログラムで使用する変数を定義します。

\begin{itemize}
\item 自動車の色の配列
\item 各自動車注文の構造体
\item トランスミッションの種類を記述する列挙型
\end{itemize}

\begin{enumerate}
\item \texttt{main} 関数の次のコード ブロックを既存のコードに追加します。 新しいコードは、ファイルの先頭または末尾に追加できます。

\begin{lstlisting}[numbers=none]
fn main() {
    // 車の色配列の作成
    let colors = todo!("Set the enum values: 0 = Blue, 1 = Green,\\
                        2 = Red, 3 = Silver");

    // 車種と初期値を宣言する
    let mut car: Car = todo!("Create `car` as a `Car` struct");     
    let mut engine: Transmission = todo!("Declare `engine` as a \\
                   `Transmission` enum, initialize to `Manual`");
}
\end{lstlisting}

\item \texttt{colors} 配列変数の定義を完了します。 自動車は、青、緑、赤、シルバーの 4 色のいずれかにすることができます。

\item \texttt{car} 構造体および \texttt{engine} 列挙型変数の宣言構文を修正します。 \texttt{engine} 列挙型を "Manual" に初期化します。

\begin{itembox}[l]{ヒント}
両方の変数を変更可能として作成することを忘れないでください。
\end{itembox}

\item プログラムをビルドします。 次のセクションに進む前に、コードがエラーなしでコンパイルされることを確認してください。
\end{enumerate}


\subsubsection{自動車注文を遂行する関数を呼び出す}

\texttt{main} 関数で、自動車注文を遂行する \texttt{car\_factory} 関数を呼び出します。 関数呼び出しでは、自動車の色が必要です。 \texttt{colors} 配列にインデックスを付ける方法を修正して、実際に期待された色を渡すようにする必要があります。

\begin{enumerate}

\item \texttt{main} 関数に次のコード ブロックを追加します。 このコードは、前の手順で追加した定義の後、かつ関数の右中かっこ \texttt{\}} の前に配置します。


\begin{lstlisting}[numbers=none]
    // トランスミッション1種類につき1台、計3台ご注文ください。

    // Car order #1: New, Manual, Hard top
    car = car_factory(String::from(todo!("Index into the `colors()`\\
                                         array")), engine, true, 0);
    println!("Car order 1: {:?}, Hard top = {}, {:?}, {}, {} miles",
              car.age.0, car.roof, car.motor, car.color, car.age.1);

    // Car order #2: Used, Semi-automatic, Convertible
    engine = Transmission::SemiAuto;
    car = car_factory(String::from(todo!("Index into the `colors()`\\
                                       array")), engine, false, 100);
    println!("Car order 2: {:?}, Hard top = {}, {:?}, {}, {} miles",
             car.age.0, car.roof, car.motor, car.color, car.age.1);

    // Car order #3: Used, Automatic, Hard top
    engine = Transmission::Automatic;
    car = car_factory(String::from(todo!("Index into the `colors()`\\
                                        array")), engine, true, 200);
    println!("Car order 3: {:?}, Hard top = {}, {:?}, {}, {} miles",
              car.age.0, car.roof, car.motor, car.color, car.age.1);
\end{lstlisting}


\item \texttt{String::from} の呼び出しで \texttt{colors} 配列のインデックス付けを修正します。 \texttt{colors} 配列には 4 つの要素があるため、注文ごとに異なる色を使用するように試行します。

\begin{itembox}[l]{ヒント}
配列内の要素は、インデックス位置 0 から開始します。 最初の要素の値はインデックス位置 0 にあります。
\end{itembox}

\item プログラムをビルドします。 コードがエラーなしでコンパイルされることを確認します。 警告メッセージはすべて無視してかまいません。

\end{enumerate}

\subsubsection{プログラムの実行}

プログラムが完了すると、この例のような出力が表示されます。

\begin{lstlisting}[numbers=none]
Car order 1: New, Hard top = true, Manual, Blue, 0 miles
Car order 2: New, Hard top = false, SemiAuto, Green, 100 miles
Car order 3: New, Hard top = true, Automatic, Red, 200 miles
\end{lstlisting}

注文 2 と 3 は中古車用ですが、 \texttt{car\_quality} 関数からは現在のところ New のみが返されます。 後の演習で、条件式を使用してそれを修正し、注文を変更する方法について確認します。


\subsubsection{解決策}

この Rust Playground 内で、プログラム出力をこの演習のソリューションと比較できます。

