\subsection{はじめに}

このモジュールでは、"特性" と "ジェネリック" について説明します。これらは、"ポリモーフィズム" の概念 (同じ関数でさまざまな型を受け入れる) に対応するために Rust で使用される方法です。 さらに、これらを使用することで、まだ宣言されていない型も含めて、さまざまな型の値を処理するコードを記述できます。


\subsubsection{学習の目的}

このモジュールでは、次のことを学習します。

\begin{itemize}
\item ジェネリック型とは何か、"ラッパー" 型でそれらはどのように使用されるのか。
\item 特性とは何か、共有動作を定義する上でそれらはどのように役立つのか。
\item カスタム型の既存の特性を実装する方法。
\item 既存の型のカスタム特性を実装する方法。
\item 特性境界はジェネリック関数を記述する上でどのように役立つのか。
\item コレクションを反復処理する Iterator 特性を実装する方法。
\end{itemize}