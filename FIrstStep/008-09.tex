\subsection{知識チェック}

次の質問に答えて、学習した内容を確認してください。

\begin{enumerate}
\item Rust の特性が役に立つのはどのような場合ですか?
\begin{itemize}
\item 省略可能なパラメーターを関数または構造体で受け取る必要がある場合。
\item 具体的な値ではなく、動作の観点から関数または構造体のパラメーターを指定する必要がある場合。
\item 借用チェッカーのコンパイル時間保証を避ける必要がある場合。
\item 値がその有効期間を過ぎても、引き続き有効にしておく必要がある場合。
\end{itemize}
\item 次の関数シグネチャはどういう意味ですか? \texttt{fn show\_on\_screen<T: Display>(data: T)}
\begin{itemize}
\item \texttt{data} パラメーターは任意の型にすることができます。
\item \texttt{data} パラメーターは、必要に応じて \texttt{Display} 特性を実装する任意の型とすることができます。
\item \texttt{data} パラメーターは、 \texttt{Display} 特性を実装する型のみに制限されています。
\item \texttt{data} パラメーターは省略可能です。
\end{itemize}
\end{enumerate}









