\section{Rust プログラムを初めて作成する}

変数、データ型、関数など、Rust の概念について学習します。

\subsubsection{学習の目的}
このモジュールでは、次のことを行います。

\begin{itemize}
\item 関数、データ型、変数など、Rust 言語の主要な概念を調べる
\item テキスト、数値、ブール値、複合データの基本的な Rust 型について理解する
\item 基本的な Rust プログラムを作成し、コンパイルして、実行する
\item プログラムから出力する方法を見つける
\end{itemize}

\subsection{はじめに}

このモジュールでは、プログラミング言語の一般的な概念について学習し、Rust でそれらがどのように実装されているかを確認します。 概念は Rust に固有のものではありませんが、すべての Rust プログラムの基礎を提供します。 これらの概念について学習することで、任意のプログラミング言語での開発をサポートするための理解を深めることができます。

\subsubsection{Rust Playground}

Rust Playground は、Rust コンパイラとのブラウザー インターフェイスです。 この言語をローカル環境にインストールする前に、またはコンパイラを利用できない場合は、プレイグラウンドを使って、Rust コードを書いてみることができます。 このコース全体を通して、サンプル コードと演習へのプレイグラウンド リンクを提供します。 現在 Rust ツールチェーンを使用できない場合でも、コードを操作できます。

Rust Playground 上で実行されるすべてのコードは、ローカルの開発環境でコンパイルして実行することもできます。 お使いのコンピューターから Rust コンパイラをぜひ操作してみてください。 Rust プレイグラウンドの詳細については、「Rust とは」モジュールで学習できます。

\subsubsection{学習の目的}


このモジュールでは、次のことを行います。

\begin{itemize}
\item 関数、データ型、変数など、Rust 言語の主要な概念を調べる。
\item テキスト、数値、ブール値、複合データの基本的な Rust 型について理解する。
\item 基本的な Rust プログラムを作成し、コンパイルして、実行する。
\item プログラムから出力する方法を確認する。
\end{itemize}

 % はじめに
\subsection{Rust プログラムの基本的な構造を理解する}

このユニットでは、簡単な Rust プログラムがどのような構造になっているかを確認します。

\subsubsection{Rust の関数}

関数は、特定のタスクを実行するコードのブロックです。 プログラム内のコードをタスクに基づいてブロックに分割します。 この分割により、コードの理解と保守が容易になります。 タスクの関数を定義したら、そのタスクを実行する必要があるときに関数を呼び出すことができます。

すべての Rust プログラムには、 という名前の関数が 1 つ必要です。 \texttt{main} 関数内のコードは、常に、Rust プログラムで最初に実行されるコードです。 \texttt{main} 関数内または他の関数内から他の関数を呼び出すことができます。



\begin{lstlisting}[numbers=none]
fn main() {
    println!("Hello, world!");
}
\end{lstlisting}

Rust 内で関数を宣言するには、\texttt{fn} キーワードを使用します。 関数名の後に、関数が入力として受け取るパラメーターまたは "引数" の数をコンパイラに指示します。 引数は、かっこ \texttt{()} 内に列記します。 "関数の本体" は、関数のタスクを実行するコードであり、中かっこ 内で定義します。 関数本体の左中かっこがかっこで囲まれた引数リストの直後に位置するようにコードを書式設定することをお勧めします。

\subsubsection{コード インデント}

関数本体では、ほとんどのコード ステートメントはセミコロン \texttt{;} で終わります。 Rust により、これらのステートメントが 1 つずつ順番に処理されます。 コード ステートメントがセミコロンで終わらない場合、Rust では、開始ステートメントが完了する前に次のコード行を実行する必要があることが認識されます。

コード内の実行関係を確認できるように、インデントを使用します。 この形式により、コードがどのように整理されているかが示され、関数タスクを完了するためのステップの流れが明らかになります。 開始コード ステートメントは、左余白からスペース 4 つ分インデントされます。 コードがセミコロンで終わっていない場合、実行する次のコード行は、さらにスペース 4 つ分インデントされます。

次に例を示します。

\begin{lstlisting}[numbers=none]
fn main() { // The function declaration is not indented

    // First step in function body
        // Substep: execute before First step can be complete

    // Second step in function body
        // Substep A: execute before Second step can be complete
        // Substep B: execute before Second step can be complete
            // Sub-substep 1: execute before Substep B can be complete

    // Third step in function body, and so on...
}
\end{lstlisting}

\subsubsection{todo! マクロ}


Rust モジュールの演習を行っていると、サンプル コードで \texttt{todo!} マクロが頻繁に使われていることに気付きます。 Rust のマクロは、可変個の入力引数を受け取る関数のようなものです。 \texttt{todo!} マクロは、Rust プログラムで未完了のコードを識別するために使用されます。 このマクロは、プロトタイプを作成するときや、完了していない動作を示す場合に便利です。

演習での \texttt{todo!} マクロの使用方法の例を次に示します。



\begin{lstlisting}[numbers=none]
fn main() {
    // Display the message "Hello, world!"
    todo!("Display the message by using the println!() macro");
}
\end{lstlisting}

\texttt{todo!} マクロを使用するコードをコンパイルすると、完了した機能を見つけることが想定されている場所で、コンパイラからパニック メッセージが返されることがあります。

\begin{lstlisting}[numbers=none]
   Compiling playground v0.0.1 (/playground)
    Finished dev [unoptimized + debuginfo] target(s) in 1.50s
     Running `target/debug/playground`
thread 'main' panicked at 'not yet implemented:\\
 Display the message by using the println!() macro', src/main.rs:3:5
note: run with `RUST_BACKTRACE=1` environment variable\\
 to display a backtrace
\end{lstlisting}

\subsubsection{println! マクロ}

\texttt{main} 関数では 1 つのタスクが実行されます。 これにより、Rust 内に事前定義されている \texttt{println!} マクロが呼び出されます。 \texttt{println!} マクロには、画面または "\texttt{println!}" に表示される 1 つ以上の入力引数が必要です。 この例では、マクロに 1 つの入力引数 (テキスト文字列 "Hello, world!") を渡します。

\subsubsection{\{\} 引数の値の置換}

Rust Learn モジュールのレッスンでは、多くの場合、中かっこ \texttt{\{\}} とその他の値のインスタンスが含まれるテキスト文字列を含む引数のリストを指定して \texttt{println!} マクロを呼び出します。 \texttt{println!} マクロにより、テキスト文字列内の中かっこ \texttt{\{\}} の各インスタンスが、リスト内の次の引数の値に置き換えられます。

次に例を示します。

\begin{lstlisting}[numbers=none]
fn main() {
    // Call println! with three arguments: a string, a value, a value
    println!("The first letter of the English alphabet is\\
     {} and the last letter is {}.", 'A', 'Z');
}
\end{lstlisting}

\texttt{println!} マクロは、文字列、値、および別の値の 3 つの引数を指定して呼び出します。 マクロにより、引数が順番に処理されます。 テキスト文字列内の中かっこ \texttt{\{\}} の各インスタンスは、リスト内の次の引数の値に置き換えられます。

出力は次のようになります。

\begin{lstlisting}[numbers=none]
The first letter of the English alphabet is A and the last letter is Z.
\end{lstlisting}

\subsubsection{自分の知識をチェックする}

次の質問に答えて、学習した内容を確認してください。

\begin{enumerate}
\item Rust プログラムでは、main 関数をいくつ使用できますか?
\begin{itemize}
\item Rust プログラムでは、必要なだけいくつでも main 関数を使用できる。
\item Rust のすべての関数で、main という名前のサブ関数を使用できる。
\item すべての Rust プログラムは、main という名前の関数を 1 つだけ持つ必要がある。
\end{itemize}

\item 新しい関数を宣言するために使用する Rust キーワードは何ですか?
\begin{itemize}
\item function
\item fn
\item func
\end{itemize}

\item println! マクロのこの呼び出しからの出力はどれですか?

\texttt{println!("\{\} is a number. \{\} is a word.", 1, "Two");}
\begin{itemize}
\item 1 は数値です。 Two is a word.
\item \{\} は数値です。 \{\} is a word.
\item \{1\} は数値です。 \{"Two"\} is a word.
\end{itemize}

\end{enumerate}



 % Rust プログラムの基本的な構造を理解する
\subsection{Rust で変数を作成して使用する}

開発者は、データを操作するコンピューター プログラムを記述します。 データの収集、分析、保存、処理、共有、報告が行われます。 "変数" を使用して、コード内で後で参照できる名前付き参照にデータを格納します。

\subsubsection{変数}

Rust では、変数はキーワード \texttt{let} を使用して宣言されます。 各変数には一意の名前が付いています。 変数が宣言されている場合は、値にバインドできます。また、後でプログラム内で値をバインドすることもできます。 次のコードでは、 \texttt{a\_number} という名前の変数を宣言しています。

\begin{lstlisting}[numbers=none]
let a_number;
\end{lstlisting}

\texttt{a\_number} 変数はまだ値にバインドされていません。 このステートメントを変更して、値を変数にバインドできます。

\begin{lstlisting}[numbers=none]
let a_number = 10;
\end{lstlisting}

\begin{itembox}[l]{注意}
キーワード: 他のプログラミング言語と同様に、 や などの特定の "キーワード" は、Rust のみが使用するために予約されています。 キーワードを関数または変数の名前として使用することはできません。
\end{itembox}

別の例を見てみましょう。 次のコードでは、2 つの変数が宣言されています。 最初の変数は宣言されていますが、値にバインドされていません。 2 番目の変数は宣言されており、値にバインドされています。 プログラムの後の部分で最初の変数の値は、単語にバインドされています。 コードでは、変数の値を表示する \texttt{println!} マクロが呼び出されます。


\begin{lstlisting}[numbers=none]
// 変数の宣言
let a_number;
    
// 2つ目の変数を宣言し、値をバインドする
let a_word = "Ten";
    
// 最初の変数に値をバインドする
a_number = 10;

println!("The number is {}.", a_number);
println!("The word is {}.", a_word);
\end{lstlisting}

この例では、次の出力が出力されます。

\begin{lstlisting}[numbers=none]
The number is 10.
The word is Ten.
\end{lstlisting}

\texttt{println!} マクロを呼び出して、 \texttt{a\_number} 変数の値をバインド前に表示しようとすると、コンパイラからエラーが返されます。

このエラー メッセージは自分で、Rust Playground で確認できます。 [実行] ボタンを選択して、コードを実行します。

\subsubsection{不変と変更可能}

Rust では、変数バインドは既定で変更できません。 変数が不変の場合、値が名前にバインドされた後に、その値を変更することはできません。

たとえば、前の例の \texttt{a\_number} 変数の値を変更しようとした場合、コンパイラからエラー メッセージを受け取ります。


\begin{lstlisting}[numbers=none]
// 不変型変数の値を変更する
a_number = 15;
\end{lstlisting}

Rust プレイグラウンドでこの変更を自分で試してみて、エラー メッセージを確認できます。

値を変更するには、まず \texttt{mut} キーワードを使用して、変数バインドを変更可能にする必要があります。

\begin{lstlisting}[numbers=none]
// `mut` キーワードは、変数を変更するためのものです。
let mut a_number = 10; 
println!("The number is {}.", a_number);

// 不変型変数の値を変更する
a_number = 15;
println!("Now the number is {}.", a_number);
\end{lstlisting}

この例では、次の出力が出力されます。

\begin{lstlisting}[numbers=none]
The number is 10.
Now the number is 15.
\end{lstlisting}

このコードは、変数 \texttt{a\_number} が変更可能になったため、エラーなくコンパイルされます。

\subsubsection{変数のシャドウ処理}

既存の変数の名前を使用する新しい変数を宣言できます。 新しい宣言によって新しいバインドが作成されます。 Rust では、新しい変数によって、前の変数がシャドウされるため、この操作は "シャドウ処理" と呼ばれます。 古い変数は引き続き存在していますが、このスコープでは参照できなくなります。

シャドウ処理を使用したコードの例を次に示します。 \texttt{shadow\_num} という名前の変数を宣言します。 変数は変更可能として定義しません。各 \texttt{let} 操作では、前の変数のバインドをシャドウ処理するときに \texttt{shadow\_num} という新しい変数が作成されるためです。


\begin{lstlisting}[numbers=none]
// 最初の変数バインディングを名前 "shadow_num" で宣言します。
let shadow_num = 5;

// 2つ目の変数バインディングを宣言し、既存の変数 "shadow_num" をシャドウします。
let shadow_num = shadow_num + 5; 

// 3つ目の変数バインディングを宣言し、変数 "shadow_num" の
// 2つ目のバインディングをシャドウする。
let shadow_num = shadow_num * 2; 

println!("The number is {}.", shadow_num);
\end{lstlisting}

出力を推測できますか? Rust Playground にアクセスしてこの例を実行してください。

\subsubsection{自分の知識をチェックする}

次の質問に答えて、学習した内容を確認してください。 


\begin{enumerate}
\item 変数の宣言と値のバインドの両方が行われている Rust ステートメントはどれですか?
\begin{itemize}
\item \texttt{let continents = 7;}

\item \texttt{continents = 7;}

\item \texttt{let continents;}
\end{itemize}


変数の値を変更可能にするために使用される Rust キーワードはどれですか?
\begin{itemize}
\item \texttt{mutable}

\item \texttt{immutable}

\item \texttt{mut}
\end{itemize}
\end{enumerate}
 % Rust で変数を作成して使用する
\subsection{数値、テキスト、true/false 値のデータ型を調べる}

Rust は静的型指定の言語です。 コンパイラは、プログラムをコンパイルして実行するために、コード内のすべての変数の正確なデータ型を認識している必要があります。 通常、コンパイラでは、バインドされた値に基づいて変数のデータ型を推測できます。 常にコード内で型を明示的に指定する必要はありません。 多くの型が可能な場合は、 型の注釈 を使用して、コンパイラに特定の型を通知する必要があります。

次の例では、 \texttt{number} 変数を 32 ビット整数として作成するようにコンパイラに指示しています。 変数名の後にデータ型 \texttt{u32} を指定します。 変数名の後にコロン \texttt{:} が使用されていることに注意してください。


\begin{lstlisting}[numbers=none]
let number: u32 = 14;
println!("The number is {}.", number);
\end{lstlisting}

変数の値を二重引用符で囲むと、コンパイラは値を数値ではなくテキストとして解釈します。 値の推定データ型が変数に指定された \texttt{u32} データ型と一致しないため、コンパイラによってエラーが発行されます。

\begin{lstlisting}[numbers=none]
let number: u32 = "14";
\end{lstlisting}

コンパイラ エラー:

\begin{lstlisting}[numbers=none]
   Compiling playground v0.0.1 (/playground)
error[E0308]: mismatched types
 --> src/main.rs:2:23
  |
2 |     let number: u32 = "14";
  |                 ---   ^^^^ expected `u32`, found `&str`
  |                 |
  |                 expected due to this

error: aborting due to previous error
\end{lstlisting}

この Rust Playground で、前述のコードを操作できます。

\subsubsection{組み込みのデータ型}

Rust には、数値、テキスト、真実性を表すいくつかの組み込みのプリミティブ データ型が用意されています。 これらの型のいくつかは単一の値を表すため、"スカラー" と呼ばれます。

\begin{itemize}
\item 整数
\item 浮動小数点数
\item ブール値
\item 文字
\end{itemize}

Rust には、文字列やタプルの値など、データ系列を操作するためのより複雑なデータ型も用意されています。

\subsubsection{数値: 整数と浮動小数点値}

Rust の整数は、ビット サイズと 符号付き プロパティによって識別できます。 \textbf{符号付き} 整数には、正または負の数値を指定できます。 \textbf{符号なし} 整数には、正の数値のみを指定できます。

\begin{tabular}{lll}
長さ & 符号付き & 符号なし\\ \hline
8 ビット & \texttt{i8} & \texttt{u8}\\ \hline
16 ビット & \texttt{i16} & \texttt{u16}\\ \hline
32 ビット & \texttt{i32} & \texttt{u32}\\ \hline
64 ビット & \texttt{i64} & \texttt{u64}\\ \hline
128 ビット & \texttt{i128} & \texttt{u128}\\ \hline
architecture-dependent & \texttt{isize} & \texttt{usize}\\
\end{tabular}

\texttt{isize} 型と \texttt{usize} 型は、プログラムが実行されているコンピューターの種類によって異なります。 64 ビット アーキテクチャでは 64 ビット型が、32 ビットアーキテクチャでは 32 ビット型が使用されます。 整数の型を指定せず、システムが型を推論できない場合は、既定で \texttt{i32} 型 (32 ビット符号付き整数) が割り当てられます。

Rust には、10 進値 \texttt{f32} (32 ビット) と \texttt{f64} (64 ビット) の 2 つの浮動小数点データ型があります。 既定の浮動小数点型は \texttt{f64} です。 最新の CPU では、\texttt{f64} 型は \texttt{f32} 型とほぼ同じ速度ですが、精度が高くなります。



\begin{lstlisting}[numbers=none]
let number_64 = 4.0;      // コンパイラはデフォルトの型である
                          // f64 を使用するように値を推論します。
let number_32: f32 = 5.0; // アノテーションで指定されたf32型
\end{lstlisting}

Rust のすべてのプリミティブ数値型では、加算、減算、乗算、除算などの算術演算がサポートされています。


\begin{lstlisting}[numbers=none]
// 加算・減算・乗算
println!("1 + 2 = {} and 8 - 5 = {} and 15 * 3 = {}", 1u32 + 2, 8i32 - 5, 15 * 3);

// 整数と浮動小数点の除算
println!("9 / 2 = {} but 9.0 / 2.0 = {}", 9u32 / 2, 9.0 / 2.0);
\end{lstlisting}

\begin{itembox}[l]{注意}
\texttt{println} 関数を呼び出すときに、データ型についての情報を Rust に通知するために、各リテラル数値にデータ型のサフィックスを追加します。 構文 \texttt{1u32} は、値が数字の 1 であることをコンパイラに伝え、値を符号なし 32 ビット整数として解釈します。   型の注釈を提供しない場合は、Rust によってコンテキストから型が推測されます。 コンテキストがあいまいである場合は、既定で \texttt{i32} 型 (32 ビット符号付き整数) が割り当てられます。
\end{itembox}

この例を Rust Playground で実行してみることができます。

\subsubsection{ブール値: True または False}

Rust のブール型は、真偽を格納するために使用されます。 \texttt{bool} 型には、 \texttt{true} または \texttt{false} の 2 つの有効な値があります。 ブール値は、条件式で広く使用されます。 \texttt{bool} ステートメントまたは値が \texttt{true} の場合は、このアクションを実行します。それ以外の場合 (ステートメントまたは値が \texttt{false} の場合) は、別のアクションを実行します。 ブール値は、多くの場合、比較チェックによって返されます。

次の例では、より大きい \texttt{>} 演算子を使用して 2 つの値をテストします。 演算子は、テストの結果を示すブール値を返します。

\begin{lstlisting}[numbers=none]
// "より大きい "テストの結果を格納する変数を宣言する。 1 > 4か?-- false
let is_bigger = 1 > 4;
println!("Is 1 > 4? {}", is_bigger);
\end{lstlisting}

\subsubsection{テキスト: 文字と文字列}

Rust では、2 つの基本的な文字列型と 1 つの文字型を持つテキスト値がサポートされています。 文字は 1 つの項目で、文字列は一連の文字です。 すべてのテキスト型は有効な UTF-8 表現です。

\texttt{char} 型は、最もプリミティブなテキスト型です。 値は、項目を単一引用符で囲んで指定します。


\begin{lstlisting}[numbers=none]
let uppercase_s = 'S';
let lowercase_f = 'f';
let smiley_face = '😃';
\end{lstlisting}

\begin{itembox}[l]{注意}
一部の言語では、 \texttt{char} 型を 8 ビット符号なし整数 (Rust の \texttt{u8} 型と同等) として扱います。 Rust の \texttt{char} 型には unicode コード ポイントが含まれていますが、utf-8 エンコードは使用しません。 Rust の \texttt{char} は、32 ビット幅に埋め込まれる 21 ビットの整数です。 \texttt{char} には、プレーン コード ポイント値が直接含まれています。
\end{itembox}

\subsubsection{文字列}

\texttt{str} 型は "文字列スライス" とも呼ばれ、文字列データであることが わかります。 ほとんどの場合、アンパサンドが型の前にある参照スタイルの構文 \texttt{&str} を使用して、これらの型を参照します。 参照については後述のモジュールで説明します。 ここでは、\texttt{&str} を、変更できない文字列データへのポインターとして考えることができます。 文字列リテラルはすべて型 \texttt{&str} です。

文字列リテラルは、初歩的な Rust の例で使用するのは便利ですが、テキストを使用するすべての状況に適しているわけではありません。 コンパイル時にすべての文字列がわかっているわけではありません。 例は、実行時にユーザーがプログラムと対話し、ターミナル経由でテキストを送信する場合です。

これらのシナリオでは、Rust には \texttt{String} という名前の 2 番目の文字列型があります。 この型は、ヒープに割り当てられます。 \texttt{String} 型を使用する場合、コードをコンパイルする前に、文字列の長さ (文字数) を把握しておく必要はありません。

\begin{itembox}[l]{実質注意的に一定時間}
ガベージ コレクションされた言語に慣れている場合は、Rust に 2 つの文字列型がある理由を疑問に思う可能性があります。{}^1 文字列はとても複雑なデータ型です。 ほとんどの言語では、それぞれのガベージ コレクターを使用してこの複雑さを解消しています。 システムの言語である Rust では、文字列に固有の複雑さがいくつか見られます。 この複雑さが加わることで、プログラムでのメモリの使用方法を非常に細かく制御できるようになります。

{}^1 実際には、Rust には 2 つより多くの文字列型があります。 このモジュールでは、 \texttt{String} 型と \texttt{&str} 型のみを扱います。 提供される文字列型について詳しくは、Rust のドキュメントをご覧ください。
\end{itembox}

Rust の所有権および借用システムについて学習するまで、 \texttt{String} と \texttt{&str} との違いについて完全に理解することはできません。 それまでは、 \texttt{String} 型データを、プログラムの実行時に変更される可能性があるテキスト データとして考えることができます。 \texttt{&str} 参照は、プログラムの実行時に変更されないテキスト データの不変ビューとなります。

\subsubsection{テキストの例}

次の例は、Rust での \texttt{char} および \texttt{&str} データ型の使用方法を示しています。

\begin{itemize}
\item \texttt{: char} 注釈構文を使用して、2 つの文字変数が宣言されます。 値は、単一引用符を使用して指定されます。
\item 3 番目の文字変数が宣言され、1 つのイメージにバインドされます。 この変数については、コンパイラにデータ型を推測させます。
\item 2 つの文字列変数が宣言され、それぞれの値にバインドされます。 文字列は二重引用符で囲まれます。
\item 文字列変数の 1 つが、データ型を指定するための \texttt{: &str} 注釈構文で宣言されています。 他の変数のデータ型は指定されていません。 コンパイラによって、この変数のデータ型がコンテキストに基づいて推測されます。
\end{itemize}

\texttt{string_1} 変数には、一連の文字の末尾に空のスペースが含まれていることに注意してください。

\begin{lstlisting}[numbers=none]
// データ型 "char "を指定
let character_1: char = 'S';
let character_2: char = 'f';
   
// コンパイラは引用符で囲まれた1つの項目を "char "データ型として解釈する
let smiley_face = '😃';

// コンパイラは引用符で囲まれた一連の項目を
// 「str」データ型として解釈し、「&str」参照を作成する
let string_1 = "miley ";

// データ型 "str "を参照構文"&str "で指定する。
let string_2: &str = "ace";

println!("{} is a {}{}{}{}.", smiley_face, character_1, string_1, character_2, string_2);
\end{lstlisting}

この例の出力を次に示します。


\begin{lstlisting}[numbers=none]
😃 is a Smiley face.
\end{lstlisting}

この例の \texttt{str} の前にアンパサンド \texttt{&} を指定しない場合は、どうなるでしょうか。 それを調べるには、この例を Rust Playground 内で実行してみます。


\subsubsection{自分の知識をチェックする}

次の質問に答えて、学習した内容を確認してください。

\begin{enumerate}
\item Rust での整数値の定義方法についての説明はどれですか?

\begin{itemize}
\item Rust の整数は、主に、8 ビット、16 ビットなどのビット サイズによって識別される。

\item Rust の整数は、ビット サイズと 符号付き プロパティによって識別される。

\item Rust の正または負の整数は、符号なし (\texttt{u}) または符号付き (\texttt{i}) の値として定義できます。
\end{itemize}

\item Rust でのテキスト文字値のサポート方法の正しい説明はどれですか?

\begin{itemize}
\item Rust には 1 つのデータ型があり、1 つの文字と複数文字のテキスト文字列の両方に使用できる。

\item 文字 (\texttt{char}) は、"A" や "z" のような 1 つのアルファベット文字にのみ使用できる。 文字列は、文字、数字、イメージなど、一連の任意の文字に使用できる。

\item Rust では、すべてのテキスト型は有効な UTF-8 表現である。
\end{itemize}

\end{enumerate}

 % 数値、テキスト、true/false 値のデータ型を調べる
\subsection{タプルと構造体を使用してデータのコレクションを定義する}

このユニットでは、タプルと構造体という、データ コレクションまたは複合データの操作に役立つ 2 つのデータ型について説明します。

\subsubsection{タプル}

タプルは、1 つの複合値に収集されたさまざまな型の値をグループ化したものです。 タプル内の個々の値は、"要素" と呼ばれます。 値は、かっこ \texttt{(<value>, <value>, ...)} で囲まれたコンマ区切りのリストとして指定されます。

タプルの長さは固定で、要素の数と同じになります。 タプルが宣言された後に、サイズを拡大または縮小することはできません。 要素を追加または削除することはできません。 タプルのデータ型は、要素のデータ型のシーケンスによって定義されます。

\subsubsection{タプルを定義する}

3 つの要素を持つタプルの例を次に示します。


\begin{lstlisting}[numbers=none]
// 長さ3のタプル
let tuple_e = ('E', 5i32, true);
\end{lstlisting}

次の表は、タプル内の各要素の値、データ型、およびインデックスを示しています。

\begin{tabular}{lll}
要素 & 値 & データ型\\ \hline
0 & E & \texttt{char}\\ \hline
1 & 5 & \texttt{i32}\\ \hline
2 & true & \texttt{bool}\\
\end{tabular}

このタプルの型シグネチャは、3 つの要素の型のシーケンス \texttt{(char, i32, bool)} によって定義されます。

\subsubsection{タプル内の要素にアクセスする}

タプル内の要素には、0 から始まるインデックス位置によってアクセスできます。 このプロセスは、"タプルのインデックス付け" と呼ばれます。 タプル内の要素にアクセスするには、構文 \texttt{<tuple>.<index>} を使用します。

次の例は、インデックスを使用してタプル内の要素にアクセスする方法を示しています。



\begin{lstlisting}[numbers=none]
// 3つの要素からなるタプルの宣言
let tuple_e = ('E', 5i32, true);

// タプルインデックスを使用し、タプル内の要素の値を表示する
println!("Is '{}' the {}th letter of the alphabet? {}", \\
         tuple_e.0, tuple_e.1, tuple_e.2);
\end{lstlisting}

例では、次の出力が表示されます。

\begin{lstlisting}[numbers=none]
Is 'E' the 5th letter of the alphabet? true
\end{lstlisting}

この例を Rust Playground 内で実行することができます。

タプルは、異なる型を 1 つの値に組み合わせる場合に便利です。 タプルでは任意の数の値を保持できるため、関数でタプルを使用して、複数の値を返すことができます。

\subsubsection{構造体}

構造体は、他の型で構成される型です。 構造体の要素は "フィールド" と呼ばれます。 タプルと同様に、構造体のフィールドは異なるデータ型を持つことができます。 構造体型の大きな利点は、各フィールドに名前を指定して値の意味を明確にできることです。

Rust プログラムで構造体を操作するには、まず構造体を名前で定義し、各フィールドのデータ型を指定します。 次に、別の名前を使用して構造体の "インスタンス" を作成します。 インスタンスを宣言する場合は、フィールドに特定の値を指定します。

Rust では、従来の構造体、タプル構造体、ユニット構造体という 3 つの構造体型がサポートされています。 これらの構造体型により、データのグループ化や操作を行うさまざまな方法がサポートされます。

\begin{itemize}
\item \textbf{従来の C 構造体}は最もよく使われています。 構造体内の各フィールドには、名前とデータ型があります。 従来の構造体を定義した後は、構文 \texttt{<struct>.<field>} を使用して構造体内のフィールドにアクセスできます。
\item \textbf{タプル構造体}は従来の構造体に似ていますが、フィールドには名前がありません。 タプル構造体内のフィールドにアクセスするには、タプルのインデックス付けの場合と同じ構文 \texttt{(<tuple>.<index>)} を使用します。 タプルの場合と同様に、タプル構造体のインデックス値は 0 から始まります。
\item \texttt{ユニット構造体}、マーカーとして最もよく使用されます。 Rust の "特徴" について学習するときに、ユニット構造体が役立つ場合がある理由について詳しく学習します。
\end{itemize}

次のコードは、3 種類の構造体型の定義例を示しています。

\begin{lstlisting}[numbers=none]
// 名前付きフィールドを持つ古典的な構造体
struct Student { name: String, level: u8, remote: bool }

// データ型のみを持つタプル構造体
struct Grades(char, char, char, char, f32);

// ユニット構造体
struct Unit;
\end{lstlisting}

\subsubsection{構造体を定義する}

\paragraph{従来の構造体}

関数と同様に、従来の構造体の本体は中かっこ \texttt{\{\}} 内で定義されます。 従来の構造体の各フィールドには、構造体内で一意の名前が付けられます。 各フィールドの型は、構文 \texttt{: <type>} で指定します。 従来の構造体内のフィールドは、コンマ区切りのリスト \texttt{<field>, <field>, ...} として指定されます。 従来の構造体の定義は、セミコロンで終了しません。


\begin{lstlisting}[numbers=none]
// 名前付きフィールドを持つ古典的な構造体
struct Student { name: String, level: u8, remote: bool }
\end{lstlisting}

従来の構造体の定義の利点は、構造体のフィールドの値に名前でアクセスできることです。 フィールド値にアクセスするには、構文 \texttt{<struct>.<field>} を使用します。

\paragraph{タプル構造体}

タプルと同様に、タプル構造体の本体はかっこ \texttt{()} 内に定義されます。 かっこは、構造体名の直後に続きます。 構造体名と左かっこの間にスペースはありません。

タプルとは異なり、タプル構造体の定義には、各フィールドのデータ型のみが含まれます。 タプル構造体のデータ型は、コンマ区切りのリスト \texttt{<type>, <type>, ...} として指定されます。

\begin{lstlisting}[numbers=none]
// データ型のみを持つタプル構造体
struct Grades(char, char, char, char, f32);
\end{lstlisting}

\subsubsection{構造体のインスタンス化}

構造体型を定義したら、型のインスタンスを作成し、各フィールドの値を指定することによって、構造体を使用します。 フィールドの値を設定するときに、フィールドを定義されている順序で指定する必要はありません。

次の例では、Student および Grades 構造体型に対して作成した定義を使用します。

\begin{lstlisting}[numbers=none]
// 古典的な構造体をインスタンス化し、
// フィールドをランダムな順序、あるいは指定した順序で指定する。
let user_1 = Student { name: String::from("Constance Sharma"),\\
 remote: true, level: 2 };
let user_2 = Student { name: String::from("Dyson Tan"),\\
 level: 5, remote: false };

// タプル構造体のインスタンスを作成し、定義された型と同じ順序で値を渡す
let mark_1 = Grades('A', 'A', 'B', 'A', 3.75);
let mark_2 = Grades('B', 'A', 'A', 'C', 3.25);

println!("{}, level {}. Remote: {}. Grades: {}, {}, {}, {}. Average: {}", 
         user_1.name, user_1.level, user_1.remote, mark_1.0, mark_1.1,\\
         mark_1.2, mark_1.3, mark_1.4);
println!("{}, level {}. Remote: {}. Grades: {}, {}, {}, {}. Average: {}", 
         user_2.name, user_2.level, user_2.remote, mark_2.0, mark_2.1,\\
         mark_2.2, mark_2.3, mark_2.4);
\end{lstlisting}

\subsubsection{文字列リテラルを String 型に変換する}

構造体やベクターなど、別のデータ構造内に格納される文字列データは、文字列リテラルの参照 (\texttt{\&str}) から \texttt{String} 型に変換する必要があります。 変換を行うには、標準の \texttt{String::from(\&str)} メソッドを使用します。 この例でのこのメソッドの使用方法に注意してください。

\begin{lstlisting}[numbers=none]
// 名前付きフィールドを持つ古典的な構造体
struct Student { name: String, level: u8, remote: bool }
...
let user_2 = Student { name: String::from("Dyson Tan"),\\
 level: 5, remote: false };
\end{lstlisting}

値を代入する前に型を変換しないと、コンパイラでエラーが発生します。

\begin{lstlisting}[numbers=none]
error[E0308]: mismatched types
  --> src/main.rs:24:15
   |
24 |         name: "Dyson Tan",
   |               ^^^^^^^^^^^
   |               |
   |               expected struct `String`, found `&str`
   |               help: try using a conversion method: `"Dyson Tan".to_string()`

error: aborting due to previous error
\end{lstlisting}

コンパイラでは、\texttt{.to\_string()} 関数を使って変換できることが示されています。 この例では、\texttt{String::from(\&str)} メソッドを使います。

この Rust Playground 内で、コード例を操作できます。

\subsubsection{自分の知識をチェックする}

次の質問に答えて、学習した内容を確認してください。


\begin{enumerate}
\item Rust の tuple とは何ですか?

\begin{itemize}
\item タプルは、異なる型の値のコレクションである。 データ型はその要素のデータ型に基づいており、長さは要素の数に基づいて固定される。

\item タプルは、異なる型の値のコレクションである。 データ型は、その要素のデータ型に基づいている。 長さは、要素の追加または削除に合わせて、伸びたり縮んだりできる。

\item タプルは、同じデータ型の値のコレクションである。 タプルのすべての要素は、同じデータ型である必要がある。 タプルの長さは、要素の数に基づいて固定される。
\end{itemize}

\item Rust での従来の構造体とタプル構造体の主な違いは何ですか?

\begin{itemize}
\item 従来の構造体のすべてのフィールドは、同じデータ型である必要がある。 タプル構造体のフィールドは、異なるデータ型でもかまわない。

\item タプル構造体の値には、インデックスを使用してアクセスできる。 従来の構造体の値には、フィールド名によってのみアクセスできる。

\item 従来の構造体の各フィールドには、名前とデータ型がある。 タプル構造体のフィールドには、名前がない。
\end{itemize}

\end{enumerate}
 % タプルと構造体を使用してデータのコレクションを定義する
\subsection{複合データに列挙型バリアントを使用する}

列挙型は、複数のバリアントのいずれも指定できる型です。 Rust で列挙型と呼ぶものは、より一般的には代数的データ型として知られています。 重要な詳細は、各列挙型バリアントには、それに付随するデータを指定できることです。

列挙型の作成には \texttt{enum} キーワードを使用します。これには、列挙型バリアントを任意に組み合わせることができます。 列挙型バリアントは、構造体と同様に、名前があるフィールドを持つことが可能です。しかし、名前がないフィールドを持つことも、まったくフィールドを持たないことも可能です。 構造体型と同様に、列挙型も大文字になります。


\subsubsection{列挙型の定義}

次の例では、Web イベントを分類する列挙型を定義します。 列挙型内の各バリアントは独立しており、さまざまな量と型の値を格納します。



\begin{lstlisting}[numbers=none]
enum WebEvent {
    // enumバリアントは、フィールドやデータ型のないユニット構造体の
    // ようにすることができます。
    WELoad,
    // enumのバリアントは、データ型を持つタプル構造体のようなもの
    // ですが、名前付きフィールドはありません。
    WEKeys(String, char),
    // enumバリアントは、フィールドとそのデータ型に名前を付けた
    // 古典的な構造体のようにすることができます。
    WEClick { x: i64, y: i64 }
}
\end{lstlisting}

この例の列挙型には、型が異なる 3 つのバリアントがあります。

\begin{itemize}
\item \texttt{WELoad} には、データ型またはデータが関連付けられていません。
\item \texttt{WEKeys} には、データ型が \texttt{String} および \texttt{char} である 2 つのフィールドがあります。
\item \texttt{WEMClick} には、名前付きフィールド \texttt{x} と \texttt{y}、およびそれらのデータ型 (\texttt{i64}) を持つ匿名構造体が含まれています。
\end{itemize}

さまざまな種類の構造体型を定義するのと同様の方法で、バリアントのある列挙型を定義します。 すべてのバリアントは、同じ \texttt{WebEvent} 列挙型でグループ化されます。 列挙型の各バリアントは、独自の型ではありません。 \texttt{WebEvent} 列挙型のバリアントを使用する関数はすべて、列挙型内のすべてのバリアントを受け入れる必要があります。 \texttt{WEClick} バリアントだけを受け入れ、他のバリアントは受け入れない関数を使用することはできません。

\subsubsection{構造体を使用して列挙型を定義する}

列挙型のバリアント要件を回避する方法は、列挙型のバリアントごとに個別の構造体を定義することです。 次に、列挙型の各バリアントでは、対応する構造体が使用されます。 構造体には、対応する列挙型バリアントによって保持されていたものと同じデータが保持されます。 この定義スタイルにより、それぞれの論理バリアントを単独で参照できるようになります。

次のコードは、この代替定義スタイルを使用する方法を示しています。 構造体は、データを保持するように定義されています。 列挙型のバリアントは、構造体を参照するように定義されています。

\begin{lstlisting}[numbers=none]
// タプル構造体を定義する
struct KeyPress(String, char);

// 古典的な構造体を定義する
struct MouseClick { x: i64, y: i64 }

// 新しい構造体のデータを使用するために enum バリアントを再定義します。
// ページの Load バリアントが boolean 型になるように更新されました。
enum WebEvent { WELoad(bool), WEClick(MouseClick), WEKeys(KeyPress) }
\end{lstlisting}

\subsubsection{列挙型をインスタンス化する}

次に、列挙型バリアントのインスタンスを作成するコードを追加しましょう。 バリアントごとに、\texttt{let} キーワードを使用して代入を行います。 列挙型の定義で特定のバリアントにアクセスするには、二重コロン \texttt{::} を含む構文 \texttt{<enum>::<variant>} を使用します。

\subsubsection{単純なバリアント: WELoad(bool)}

\texttt{WebEvent} 列挙型の最初のバリアントには、\texttt{WELoad(bool)} という 1 つのブール値があります。 このバリアントは、前のユニットでブール値を操作した場合と同様の方法でインスタンス化します。

\begin{lstlisting}[numbers=none]
let we_load = WebEvent::WELoad(true);
\end{lstlisting}

\subsubsection{構造体バリアント: WEClick(MouseClick)}

2 番目のバリアントには、従来の構造体 \texttt{WEClick(MouseClick)} が含まれています。 構造体には \texttt{x} および \texttt{y} という 2 つの名前付きフィールドがあり、両方のフィールドに \texttt{i64} データ型があります。 このバリアントを作成するには、まず構造体をインスタンス化します。 次に、バリアントをインスタンス化するための呼び出しで、構造体を引数として渡します。

\begin{lstlisting}[numbers=none]
// MouseClick構造体のインスタンスを作成し、座標値をバインドする。
let click = MouseClick { x: 100, y: 250 };

// WEClickバリアントがクリック構造体のデータを使用するように設定します。
let we_click = WebEvent::WEClick(click);
\end{lstlisting}

\subsubsection{タプル バリアント: WEKeys(KeyPress)}

最後のバリアントにはタプル \texttt{WEKeys(KeyPress)} が含まれています。 タプルには、\texttt{String} および \texttt{char} データ型を使用する 2 つのフィールドがあります。 このバリアントを作成するには、まずタプルをインスタンス化します。 次に、バリアントをインスタンス化するための呼び出しで、タプルを引数として渡します。

\begin{lstlisting}[numbers=none]
// KeyPressタプルをインスタンス化し、キー値をバインドする
let keys = KeyPress(String::from("Ctrl+"), 'N');
    
// WEKeys バリアントがキータプルのデータを使用するように設定します。
let we_key = WebEvent::WEKeys(keys);
\end{lstlisting}

このコードでは、\texttt{String::from("<value>")} 構文を使用していることに注意してください。 この構文では、Rust の \texttt{from} メソッドを呼び出すことによって型 \texttt{String} の値が作成されます。 メソッドには、二重引用符で囲まれたデータの入力引数が必要です。

\subsubsection{列挙型の例}

列挙型バリアントをインスタンス化するための最終的なコードは次のようになります。

\begin{lstlisting}[numbers=none]
// タプル構造体の定義
#[derive(Debug)]
struct KeyPress(String, char);

// 古典的な構造体の定義
#[derive(Debug)]
struct MouseClick { x: i64, y: i64 }

// 構造体のデータを使用する WebEvent enum variant と
// page Load variant の boolean 型を定義する。
#[derive(Debug)]
enum WebEvent { WELoad(bool), WEClick(MouseClick), WEKeys(KeyPress) }

// MouseClick構造体のインスタンスを作成し、座標値をバインドする。
let click = MouseClick { x: 100, y: 250 };
println!("Mouse click location: {}, {}", click.x, click.y);
    
// KeyPressタプルをインスタンス化し、キー値をバインドする
let keys = KeyPress(String::from("Ctrl+"), 'N');
println!("\nKeys pressed: {}{}", keys.0, keys.1);
    
// WebEvent enum variants のインスタンス化
// ページロードのブーリアン値をtrueに設定する
let we_load = WebEvent::WELoad(true);
// WEClickバリアントがクリック構造体のデータを使用するように設定します。
let we_click = WebEvent::WEClick(click);
// WEKeys バリアントがキータプルのデータを使用するように設定します。
let we_key = WebEvent::WEKeys(keys);
    
// WebEvent enum variantsの値を表示する。
// enumの構造とデータを読みやすい形で表示するには、{:#?}構文を使用します。
println!("\nWebEvent enum structure: \n\n {:#?} \n\n {:#?} \n\n {:#?}",
         we_load, we_click, we_key);
\end{lstlisting}

Rust Playground 内で、このコード例を操作してみてください。

\subsubsection{デバッグ ステートメント}

前の例で、次のコード ステートメントを探します。 このステートメントは、コード内のいくつかの場所で使用されています。

\begin{lstlisting}[numbers=none]
// 出力されたデータを確認できるように、Debugフラグを設定する
#[derive(Debug)]
\end{lstlisting}

\texttt{\#[derive(Debug)]} 構文を使用すると、コードの実行中に、標準出力では見ることのできない特定の値を確認できます。 \texttt{println!} マクロでデバッグ データを表示するには、構文 \texttt{\{:\#?\}} を使用して、読み取り可能な方法でデータを書式設定します。

\subsubsection{自分の知識をチェックする}


次の質問に答えて、学習した内容を確認してください。


\begin{enumerate}
\item Rust の列挙型のすべてのバリアントは、同じ型にグループ化されます。 列挙型のいずれかのバリアントを使う関数では、そのすべてのバリアントを受け入れる必要があります。 列挙型バリアントのこれらの要件を回避するにはどうすればよいですか?

\begin{itemize}
\item 列挙型でバリアントごとに個別の構造体を定義して、バリアント データを格納する。

\item バリアントを 1 つだけ保持するように列挙型を定義する。

\item すべて同じ型のバリアントを使用するように、列挙型を定義する。
\end{itemize}

\end{enumerate}
 % 複合データに列挙型バリアントを使用する
\subsection{Rust で関数を操作する}

関数は、Rust 内でコードを実行する主要な方法です。 この言語で、最も重要な関数の 1 つは既に見た \texttt{main} 関数です。 このユニットでは、関数を定義する方法を詳細に説明します。

\subsubsection{関数を定義する}

Rust での関数定義は、\texttt{fn} キーワードで始まります。 関数名の後に、関数の入力引数を、かっこ内のデータ型のコンマ区切りリストとして指定します。 中かっこは、関数本体の開始と終了の位置をコンパイラに伝えます。


\begin{lstlisting}[numbers=none]
fn main() {
    println!("Hello, world!");
    goodbye();
}

fn goodbye() {
    println!("Goodbye.");
}
\end{lstlisting}

関数を呼び出すには、その名前をかっこ内の入力引数と共に使用します。 関数に入力引数が存在しない場合は、かっこを空のままにします。 この例では、 \texttt{main} と \texttt{goodbye} の両方の関数に入力引数がありません。

\texttt{main} 関数の後に \texttt{goodbye} 関数が定義されていることにお気付きかと思います。 \texttt{goodbye} 関数は、 \texttt{main} を定義する前に定義しました。 Rust では、関数がファイル内のどこかに定義されている限り、ファイル内のどこに定義されているかは留意されません。

\subsubsection{入力引数を渡す}

関数に入力引数がある場合は、各引数に名前を付け、関数宣言の最初にデータ型を指定します。 引数には変数のように名前が付けられているので、関数本体内で引数にアクセスできます。

\texttt{goodbye} 関数を変更し、入力引数として文字列データへのポインターを受け取るようにしてみましょう。

\begin{lstlisting}[numbers=none]
fn goodbye(message: &str) {
    println!("\n{}", message);
}

fn main() {
    let formal = "Formal: Good bye.";
    let casual = "Casual: See you later!";
    goodbye(formal);
    goodbye(casual);
}
\end{lstlisting}

2 つの異なる引数値を使用して \texttt{main} 関数からこの関数を呼び出すことでテストし、出力を調べます。

\begin{lstlisting}[numbers=none]
Formal: Good bye.
Casual: See you later!
\end{lstlisting}

\subsubsection{値を返す}

関数が値を返す場合は、関数の引数のリストの後、関数本体の左中かっこの前に構文 \texttt{-> <type>} を追加します。 矢印の構文 \texttt{->} は、関数から呼び出し元に値が返されることを示します。 コンパイラは、\texttt{<type>} の部分により、返された値のデータ型を知ることができます。

Rust での一般的な方法では、関数の最後のコード行を返す値と同じにすることで、関数の終了時に値を返します。 この動作の例を次に示します。 \texttt{divide\_by\_5} 関数からは、入力値を 5 で割った結果が呼び出し元の関数に返されます。

\begin{lstlisting}[numbers=none]
fn divide_by_5(num: u32) -> u32 {
    num / 5
}

fn main() {
    let num = 25;
    println!("{} divided by 5 = {}", num, divide_by_5(num));
}
\end{lstlisting}

出力は次のようになります。

\begin{lstlisting}[numbers=none]
25 divided by 5 = 5
\end{lstlisting}

関数内の任意の場所で \texttt{return} キーワードを使って、実行を停止し、呼び出し元に値を返すことができます。 通常、 \texttt{return} キーワードは、条件テストと組み合わせて使います。

次に、 \texttt{num} の値が 0 の場合は \texttt{return} キーワードを明示的に使用して関数から早期に戻る例を示します。

\begin{lstlisting}[numbers=none]
fn divide_by_5(num: u32) -> u32 {
    if num == 0 {
        // Return early
        return 0;
    }
    num / 5
}
\end{lstlisting}

\texttt{return} キーワードを明示的に使用する場合は、セミコロンでステートメントを終了します。 \texttt{return} キーワードを使用せずに戻り値を返す場合は、ステートメントをセミコロンで終了しないでください。 戻り値 \texttt{num / 5} のステートメントで、最後にセミコロンを使用しなかったことに気付かれたかもしれません。

\subsubsection{シグネチャを確認する}

関数の宣言の最初の部分は、"関数シグネチャ" と呼ばれます。

この例の \texttt{goodbye} 関数のシグネチャには、これらの特性があります。

\begin{itemize}
\item \texttt{fn}: Rust の関数宣言キーワード。
\item \texttt{goodbye}: 関数名。
\item \texttt{(message: \&str)}: 関数の引数または "\texttt{(message: \&str)}" リスト。 入力値としては、文字列データへの 1 つのポインターが必要です。
\item \texttt{-> bool}: 矢印は、この関数が常に返す値の型を指しています。
\end{itemize}

\texttt{goodbye} 関数は、入力として 1 つの文字列ポインターを受け取り、ブール値を出力します。

この Rust Playground 内で、コード例を操作できます。

 % Rust で関数を操作する
\subsection{演習: 車を作成する関数を記述する}

この演習では、列挙型、構造体、および関数を使用して、新車の注文を処理します。 課題は、コンパイルして実行できるようにサンプル コードを修正することです。

この演習用のサンプル コードを操作するには、次の 2 つの方法があります。

\begin{itemize}
\item コードをコピーし、ローカルの開発環境で編集します。
\item 準備済みの Rust Playground 内でコードを開きます。
\end{itemize}

\begin{itembox}[l]{注意}
サンプル コードで、\texttt{todo!} マクロを探します。 このマクロは、完了または更新する必要があるコードを示しています。
\end{itembox}

\subsubsection{列挙型の定義}

最初のタスクでは、列挙型の定義で構文の問題を修正して、コードをコンパイルします。

\begin{enumerate}

\item サンプル コードの最初のブロックを開きます。

次のコードをコピーしてローカルの開発環境で編集するか、この用意されている Rust プレイグラウンドでコードを開きます。


\begin{lstlisting}[numbers=none]
// Car構造体を宣言し、4つの名前付きフィールドで車両を記述する。
struct Car {
    color: String,
    transmission: Transmission,
    convertible: bool,
    mileage: u32,
}

#[derive(PartialEq, Debug)]
// Carトランスミッションの種類を表すenumを宣言
enum Transmission {
    // todo!("Fix enum definition so code compiles");
    Manual;
    SemiAuto;
    Automatic;
}
\end{lstlisting}

\item プログラムが正常にコンパイルされるように、 \texttt{Transmission} 列挙型内の構文エラーを修正します。

次のセクションに進む前に、コードがコンパイルされることを確認してください。 コードではまだ出力は表示されませんが、エラーなしでコンパイルされる必要があります。

コンパイラからの "警告" メッセージは無視してかまいません。 警告は、列挙型と構造体の定義を宣言したものの、これらをまだ使用していないことが原因で生成されます。

\end{enumerate}

\subsubsection{構造体のインスタンス化}

次に、 \texttt{car\_factory} 関数のコードを追加して、 \texttt{Car} 構造体のインスタンスを作成します。 入力引数の値を使用して、自動車の特性を割り当てます。

\begin{enumerate}
\item 既存のコードに次のコード ブロックを追加します。 新しいコードは、ファイルの先頭または末尾に追加できます。


\begin{lstlisting}[numbers=none]
// 入力引数の値を使って「クルマ」を作る
// - Color of car (String)
// - Transmission type (enum value)
// - Convertible (boolean, true if car is a convertible)
fn car_factory(color: String, transmission: Transmission,\\
                                       convertible: bool) {

    // 入力された引数の値を使用する
    // すべての新車は常に走行距離ゼロ
    let car: Car = todo!("Create an instance of a `Car` struct");
}
\end{lstlisting}

\item コードをリビルドし、コンパイルされることを確認します。 ここでも、警告メッセージはすべて無視してかまいません。

\item \texttt{car} 変数の宣言を完了して、"Car" 構造体のインスタンスを作成します。 新車では、関数に渡された入力引数の値を使用する必要があります。 すべての新車の走行距離はゼロです。

\begin{itembox}[l]{ヒント}
ステートメントを型宣言 \texttt{let car: Car} からインスタンス化 \texttt{let car = Car \{ ... \}} に変更する必要があります。
\end{itembox}

\item コードをリビルドし、コンパイルされることを確認します。
\end{enumerate}

\subsubsection{関数から値を返す}

ここで、作成された \texttt{Car} 構造体を返すように \texttt{car\_factory} 関数を更新します。 値を返すには、関数シグネチャで値の型を宣言する必要があり、関数本体で値を指定する必要があります。

\begin{enumerate}
\item 関数シグネチャを変更して、戻り値の型を \texttt{Car} 構造体として宣言します。 ファイル内の次のコード行を変更します。

\begin{lstlisting}[numbers=none]
fn car_factory(color: String, transmission: Transmission,\\
   convertible: bool) = todo!("Return a `Car` struct") {
\end{lstlisting}

\begin{itembox}[l]{ヒント}
大文字と小文字の区別に注意してください。 コードはまだコンパイルしないでください。
\end{itembox}

\item 新しく作成された車を返すには、 \texttt{Car} 構造体のインスタンスを作成したステートメントを調整します。

\begin{lstlisting}[numbers=none]
    let car: Car = todo!("An instance of a `Car` struct",\\
                   "Set the function return value");
}
\end{lstlisting}

\begin{itembox}[l]{ヒント}
前のセクションでは、\texttt{Car} 構造体のインスタンスが正しく作成されるように、\texttt{let car: Car =} ステートメントを変更しました。 このステップを完了するに当たって、このコードを簡単にします。 1 つのステートメントで、 \texttt{Car} 構造体を作成し、新しく作成された車を返すことができます。 \texttt{let} または \texttt{return} キーワードを使用する必要はありません。
\end{itembox}

\item コードをリビルドし、エラーなしでコンパイルされることを確認します。
\end{enumerate}


\subsubsection{関数を呼び出す}

関数を呼び出し、自動車をビルドする準備ができました。

\begin{enumerate}
\item \texttt{main} 関数を既存のコードに追加します。 新しいコードは、ファイルの先頭または末尾に追加できます。


\begin{lstlisting}[numbers=none]
fn main() {
    // 新車3台分の注文が入りました
    // mutableなcar変数を宣言し、それを全車種に再利用することにします
    let mut car = car_factory(String::from("Red"),\\
                    Transmission::Manual, false);
    println!("Car 1 = {}, {:?} transmission, convertible: {}, mileage: {}",
       car.color, car.transmission, car.convertible, car.mileage);

    car = car_factory(String::from("Silver"), Transmission::Automatic, true);
    println!("Car 2 = {}, {:?} transmission, convertible: {}, mileage: {}",
       car.color, car.transmission, car.convertible, car.mileage);

    car = car_factory(String::from("Yellow"), Transmission::SemiAuto, false);
    println!("Car 3 = {}, {:?} transmission, convertible: {}, mileage: {}",
       car.color, car.transmission, car.convertible, car.mileage);    
}
\end{lstlisting}

\item コードをリビルドします。 宣言されたすべての項目が使用されるようになったため、コンパイラによってエラーや警告は発行されません。 次の出力が表示されます。


\begin{lstlisting}[numbers=none]
Car 1 = Red, Manual transmission, convertible: false, mileage: 0
Car 2 = Silver, Automatic transmission, convertible: true, mileage: 0
Car 3 = Yellow, SemiAuto transmission, convertible: false, mileage: 0
\end{lstlisting}

\end{enumerate}

\subsubsection{解決策}

この Rust Playground 内で、コードを準備済みソリューションと比較することができます。 % 演習: 車を作成する関数を記述する
\subsection{まとめ}

このモジュールでは、Rust プログラムの基本的な構造を確認しました。 \texttt{main} 関数は、すべての Rust プログラムへのエントリ ポイントです。 \texttt{println!} マクロを使うと、変数値を表示して、プログラムの進行状況を示すことができます。 変数は、 \texttt{let} キーワードで定義します。 \texttt{mut} キーワードを使って、それらの値を変更不可または変更可 (変更できる) として宣言できます。

多くのプライマリ データ型や複合データ型など、Rust 言語の主要な概念について調べました。 整数と浮動小数点数、文字とテキスト文字列、ブール値 true/false を使用する方法について学習しました。 Rust 言語では、データ型が厳密に解釈されます。 データ型が正しく定義および使用されている場合にのみ、プログラムは正常にコンパイルされて実行されます。

演習では、\texttt{struct} と \texttt{enum} に格納されているデータを使って、車を構築する関数を作成しました。 サンプル プログラムで \texttt{todo!} マクロのインスタンスを探し、コードを完成させました。 Rust プレイグラウンドを使用して、コードを変更し、プログラムをコンパイルし、実行可能ファイルを実行しました。

このラーニング パスの次のモジュールでは、Rust のデータ型をさらに調べ、if/else 条件式をプログラムで使用する方法を説明します。 % まとめ