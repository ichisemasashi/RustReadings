\subsection{まとめ}

このモジュールでは、Rust プログラムの基本的な構造を確認しました。 \texttt{main} 関数は、すべての Rust プログラムへのエントリ ポイントです。 \texttt{println!} マクロを使うと、変数値を表示して、プログラムの進行状況を示すことができます。 変数は、 \texttt{let} キーワードで定義します。 \texttt{mut} キーワードを使って、それらの値を変更不可または変更可 (変更できる) として宣言できます。

多くのプライマリ データ型や複合データ型など、Rust 言語の主要な概念について調べました。 整数と浮動小数点数、文字とテキスト文字列、ブール値 true/false を使用する方法について学習しました。 Rust 言語では、データ型が厳密に解釈されます。 データ型が正しく定義および使用されている場合にのみ、プログラムは正常にコンパイルされて実行されます。

演習では、\texttt{struct} と \texttt{enum} に格納されているデータを使って、車を構築する関数を作成しました。 サンプル プログラムで \texttt{todo!} マクロのインスタンスを探し、コードを完成させました。 Rust プレイグラウンドを使用して、コードを変更し、プログラムをコンパイルし、実行可能ファイルを実行しました。

このラーニング パスの次のモジュールでは、Rust のデータ型をさらに調べ、if/else 条件式をプログラムで使用する方法を説明します。