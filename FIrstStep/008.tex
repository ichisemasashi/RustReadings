\section{ジェネリック型と特性を実装する}

ジェネリック型および特性の概要と、これらを Rust で使用する方法について理解します。

\subsubsection{学習の目的}

このモジュールでは、次のことを学習します。

\begin{itemize}
\item ジェネリック型とは何か、"ラッパー" 型でそれらはどのように使用されるのか。
\item 特性とは何か、共有動作を定義する上でそれらはどのように役立つのか。
\item カスタム型の既存の特性を実装する方法。
\item 既存の型のカスタム特性を実装する方法。
\item 特性境界はジェネリック関数を記述する上でどのように役立つのか。
\item コレクションを反復処理する Iterator 特性を実装する方法。
\end{itemize}

\subsection{はじめに}

このモジュールでは、"特性" と "ジェネリック" について説明します。これらは、"ポリモーフィズム" の概念 (同じ関数でさまざまな型を受け入れる) に対応するために Rust で使用される方法です。 さらに、これらを使用することで、まだ宣言されていない型も含めて、さまざまな型の値を処理するコードを記述できます。


\subsubsection{学習の目的}

このモジュールでは、次のことを学習します。

\begin{itemize}
\item ジェネリック型とは何か、"ラッパー" 型でそれらはどのように使用されるのか。
\item 特性とは何か、共有動作を定義する上でそれらはどのように役立つのか。
\item カスタム型の既存の特性を実装する方法。
\item 既存の型のカスタム特性を実装する方法。
\item 特性境界はジェネリック関数を記述する上でどのように役立つのか。
\item コレクションを反復処理する Iterator 特性を実装する方法。
\end{itemize} % はじめに
\subsection{ジェネリック データ型とは}

ジェネリック データ型とは、その他の部分的に不明な型について定義される型です。 このコースでは最初から、多くのジェネリック データ型を使用してきました。次に例を示します。

\begin{itemize}
\item \texttt{Option<T>} 列挙型は、そのバリアント \texttt{Some} に含まれる値である型 \texttt{T} についてジェネリックです。
\item \texttt{Result<T, E>} は、その \texttt{Ok} と \texttt{Err} のバリアントにそれぞれ含まれる成功と失敗の両方の種類についてジェネリックです。
\item ベクター型 \texttt{Vec<T>}、配列型 \texttt{\[T; n\]}、ハッシュ マップ \texttt{HashMap<K, V>} は、それぞれに格納される型についてジェネリックです。
\end{itemize}

ジェネリック型を使用すると、定義された型によって保持される内部型に関して多くの懸念を抱えることなく必要な操作を指定できます。

新しいジェネリック型を実装するには、構造体の名前の直後に、型パラメーターの名前を山かっこで囲んで宣言する必要があります。 これで、構造体の定義内でジェネリック型を使用することができます。それ以外の場合は、具象データ型を指定します。


\begin{lstlisting}[numbers=none]
struct Point<T> {
    x: T,
    y: T,
}

fn main() {
    let boolean = Point { x: true, y: false };
    let integer = Point { x: 1, y: 9 };
    let float = Point { x: 1.7, y: 4.3 };
    let string_slice = Point { x: "high", y: "low" };
}
\end{lstlisting}

上記のコードでは、 \texttt{Point<T>} 構造体が定義されます。 この構造体は \texttt{x} 値と \texttt{y} 値のあらゆる型 (\texttt{T}) を保持します。

\texttt{T} により任意の具象型が想定されるものの、 \texttt{x} と \texttt{y} は同じ型であると定義されているため、同じ型とする必要があります。 次のスニペットのように、含まれている値の型がそれぞれ異なる \texttt{Point<T>} のインスタンスを作成しようとすると、コードはコンパイルされません。

\begin{lstlisting}[numbers=none]
struct Point<T> {
    x: T,
    y: T,
}

fn main() {
    let wont_work = Point { x: 25, y: true };
}
\end{lstlisting}


\begin{lstlisting}[numbers=none]
    error[E0308]: mismatched types
     --> src/main.rs:7:39
      |
    7 |     let wont_work = Point { x: 25, y: true };
      |                                       ^^^^ expected integer, found `bool`
\end{lstlisting}

エラー メッセージが表示され、 \texttt{y} フィールドに期待される型は整数であったことが示されます。 しかし、 \texttt{x} と同じ型になるように \texttt{y} を定義してあるので、コンパイラからは型の不一致エラーが報告されました。

次の例のように、複数のジェネリック型パラメーターを使用できます。 今回、 \texttt{x} と \texttt{y} が異なる型の値になれるよう、2 つの型で \texttt{Point<T, U>} ジェネリックを示します。

\begin{lstlisting}[numbers=none]
struct Point<T, U> {
    x: T,
    y: U,
}

fn main() {
    let integer_and_boolean = Point { x: 5, y: false };
    let float_and_string = Point { x: 1.0, y: "hey" };
    let integer_and_float = Point { x: 5, y: 4.0 };
    let both_integer = Point { x: 10, y: 30 };
    let both_boolean = Point { x: true, y: true };
}
\end{lstlisting}

上記のすべての \texttt{Point} 型には、それぞれ "\texttt{Point}" が使用されています。 順序は次のとおりです。

\begin{itemize}
\item \texttt{Point<integer, bool>}
\item \texttt{Point<f64, \&'static str>}
\item \texttt{Point<integer, f64>}
\item \texttt{Point<integer, integer>}
\item \texttt{Point<bool, bool>}
\end{itemize}

そのため、これらの値を互いに直接組み合わせることは、そのような対話式操作をコードに実装するまで、実際には行うことができません。

次のユニットでは、特性について学習し、コード内でジェネリック型を使用するとどのように役立つのかを明らかにします。 それらを使用すると、互いに異なるが関連している型のオブジェクトを操作するジェネリック関数を記述することができます。

 % ジェネリック データ型とは
\subsection{特性を使用して共有動作を定義する}

特性とは、型のグループで実装することができる共通のインターフェイスです。 Rust の標準ライブラリには、次のような便利な特性が多数用意されています。

\begin{itemize}
\item ソースからバイトを読み取ることができる値に対する \texttt{io::Read}。
\item バイトを書き込むことができる値に対する \texttt{io::Write}。
\item 書式指定子 "{:?}" を使用してコンソールに出力できる値に対する \texttt{Debug}。
\item メモリ内に明示的に複製できる値に対する \texttt{Clone}。
\item \texttt{String} に変換できる値に対する \texttt{ToString}。
\item 数値におけるゼロ、ベクターにおける空、 \texttt{String} における “” など、適切な既定値を持つ型に対する \texttt{Default} 。
\item 値のシーケンスを生成できる型に対する \texttt{Iterator} 。
\end{itemize}

各特性定義は、不明な型について定義されたメソッドのコレクションであり、通常は、その実装元が実行できる機能または動作を表します。

"2 次元領域がある" という概念を表すために、次の特性を定義できます。


\begin{lstlisting}[numbers=none]
trait Area {
    fn area(&self) -> f64;
}
\end{lstlisting}

ここでは、 \texttt{trait} キーワードと、その後に特性の名前 (この場合は \texttt{Area}) を使用して特性を宣言します。

中かっこ内で、この特性を実装する型の動作を記述するメソッド シグネチャを宣言します。この例では、関数シグネチャ \texttt{fn area(\&self) -> f64} です。 次に、この特性を実装する各型ではその独自のカスタム動作をメソッドの本体に指定する必要があることがコンパイラによって確認されます。

ここで、特性 \texttt{Area} を実装する新しい型をいくつか作成してみましょう。


\begin{lstlisting}[numbers=none]
struct Circle {
    radius: f64,
}

struct Rectangle {
    width: f64,
    height: f64,
}

impl Area for Circle {
    fn area(&self) -> f64 {
        use std::f64::consts::PI;
        PI * self.radius.powf(2.0)
    }
}

impl Area for Rectangle {
    fn area(&self) -> f64 {
        self.width * self.height
    }
}
\end{lstlisting}

型の特性を実装するには、キーワード \texttt{impl Trait for Type} を使用します。ここで、 \texttt{Trait} は実装される特性の名前であり、 \texttt{Type} は実装元の構造体または列挙型の名前です。

\texttt{impl} ブロック内には、特性の定義で必要とされたメソッド シグネチャを配置し、特性のメソッドで実行する特定の型を対象にした特定の動作を含むメソッド本体を入力します。

指定された特性が型によって実装されると、そのコントラクトは確実に維持されます。 特性を実装したら、通常のメソッドを呼び出すのと同じ方法で、 \texttt{Circle} と \texttt{Rectangle} のインスタンス上でメソッドを呼び出すことができます。次のようになります。

\begin{lstlisting}[numbers=none]
let circle = Circle { radius: 5.0 };
let rectangle = Rectangle {
    width: 10.0,
    height: 20.0,
};

println!("Circle area: {}", circle.area());
println!("Rectangle area: {}", rectangle.area());
\end{lstlisting}

このコードは、こちらの Rust Playground のリンクで操作できます。

 % 特性を使用して共有動作を定義する
\subsection{derive 特性を使用する}

カスタム型は、実際には少々使いにくいことにお気づきかもしれません。 このシンプルな \texttt{Point} 構造体を他の \texttt{Point} インスタンスと比較したり、ターミナルに表示したりすることはできません。 このような問題があるため、derive 属性を使用して、構造体に対して新しい項目が自動的に生成されるようにすることをお勧めします。

\subsubsection{ジェネリック型の短所}

次のコード例を見てみましょう。


\begin{lstlisting}[numbers=none]
struct Point {
    x: i32,
    y: i32,
}

fn main() {
    let p1 = Point { x: 1, y: 2 };
    let p2 = Point { x: 4, y: -3 };

    if p1 == p2 { // can't compare two Point values!
        println!("equal!");
    } else {
        println!("not equal!");
    }

    println!("{}", p1); // can't print using the
                        // '{}' format specifier!
    println!("{:?}", p1); //  can't print using the
                          // '{:?}' format specifier!

}
\end{lstlisting}

上記のコードは、3 つの理由で失敗します。 出力を次に示します。

\begin{lstlisting}[numbers=none]
    error[E0277]: `Point` doesn't implement `std::fmt::Display`
      --> src/main.rs:10:20
       |
    10 |     println!("{}", p1);
       |                    ^^ `Point` cannot be formatted with\\
                                the default formatter
       |
       = help: the trait `std::fmt::Display` is not implemented\\
         for `Point`
       = note: in format strings you may be able to use `{:?}`with\\
         (or {:#?} for pretty-print) instead
       = note: required by `std::fmt::Display::fmt`
       = note: this error originates in a macro (in Nightly builds,\\
         run with -Z macro-backtrace for more info)

    error[E0277]: `Point` doesn't implement `Debug`
      --> src/main.rs:11:22
       |
    11 |     println!("{:?}", p1);
       |                      ^^ `Point` cannot be formatted\\
                                 using `{:?}`
       |
       = help: the trait `Debug` is not implemented for `Point`
       = note: add `#[derive(Debug)]` or manually implement `Debug`
       = note: required by `std::fmt::Debug::fmt`
       = note: this error originates in a macro (in Nightly builds,\\
         run with -Z macro-backtrace for more info)

    error[E0369]: binary operation `==` cannot be applied to\\
                                        type `Point`
      --> src/main.rs:13:11
       |
    13 |     if p1 == p2 {
       |        -- ^^ -- Point
       |        |
       |        Point
       |
       = note: an implementation of `std::cmp::PartialEq` might\\
         be missing for `Point`

    error: aborting due to 3 previous errors#+end_example
\end{lstlisting}

\texttt{Point} 型によって次の特性が実装されないため、このコードはコンパイルに失敗します。
\begin{itemize}
\item \texttt{Debug} 特性。 \texttt{\{:?\}} 書式指定子を使用して型を書式設定できるようにするものであり、プログラマ向けのデバッグ コンテキストで使用されます。
\item \texttt{Display} 特性。 \texttt{\{\}} 書式指定子を使用して型を書式設定できるようにするものであり、 \texttt{Debug} に似ています。 ただし、ユーザー向けの出力には \texttt{Display} が適しています。
\item \texttt{PartialEq} 特性。実装元の同等性を比較できるようにするものです。
\end{itemize}

\subsubsection{derive を使用する}

幸いなことに、 \texttt{\#[derive(Trait)]} 属性を使用すると、その各フィールドで特性が実装される場合に、 \texttt{Debug} および \texttt{PartialEq} 特性が Rust コンパイラによって自動的に実装されるようにすることができます。

\begin{lstlisting}[numbers=none]
#[derive(Debug, PartialEq)]
struct Point {
    x: i32,
    y: i32,
}
\end{lstlisting}

コードはまだコンパイルに失敗します。エンド ユーザー向けであるという理由から \texttt{Display} 特性の自動実装が Rust の標準ライブラリに用意されていないことが原因です。 しかし、その行をコメントアウトすれば、コードによって次の出力が生成されるようになります。

\begin{lstlisting}[numbers=none]
    not equal!
    Point { x: 1, y: 2 }
\end{lstlisting}

それでも、次のように、使用する型の \texttt{Display} 特性を自分で実装することができます。

\begin{lstlisting}[numbers=none]
use std::fmt;

impl fmt::Display for Point {
    fn fmt(&self, f: &mut fmt::Formatter<'_>) -> fmt::Result {
        write!(f, "({}, {})", self.x, self.y)
    }
}
\end{lstlisting}

これで、コードは次のようにコンパイルされます。

\begin{lstlisting}[numbers=none]
    not equal!
    (1, 2)
    Point { x: 1, y: 2 }
\end{lstlisting}

この例のコードについては、こちらの Rust Playground のリンクをご覧ください。

 % derive 特性を使用する
\subsection{特性境界とジェネリック関数を使用する}

特性を使用すると、関数の定義方法次第でさまざまな型を受け取ることができます。これは、型によって特性が実装されると、その特性に従って抽象的に扱うことができるためです。

関数の引数は匿名型パラメーターとして宣言できます。この場合、呼び出し先は、匿名型パラメーターによって宣言された境界を持つ型を備えている必要があります。

Web アプリケーションを記述していて、値を JSON 形式にシリアル化するためのインターフェイスを用意する必要があるとします。 次のような特性を記述することが可能です。


\begin{lstlisting}[numbers=none]
trait AsJson {
    fn as_json(&self) -> String;
}
\end{lstlisting}

さらに、 \texttt{AsJson} 特性を実装する任意の型を受け入れる関数を記述することが可能です。 これらは、 \texttt{impl} の後に一連の特性境界を続けるようにして記述します。

\begin{lstlisting}[numbers=none]
fn send_data_as_json(value: &impl AsJson) {
    println!("Sending JSON data to server...");
    println!("-> {}", value.as_json());
    println!("Done!\n");
}
\end{lstlisting}

ここで、特性名と \texttt{impl} キーワードを指定します。 \texttt{value} パラメーターに具体的な型を使用する代わりにこれらの値で指定します。 \texttt{value} パラメーターでは、定義された特性を使用するあらゆる型が受け取られます。 関数は具象型を受け取ってもそれについて何も認識できません。そのため、使用できるメソッドは、匿名型パラメーターの特性境界で提供されるものみとなります。

使用する構文は少し異なりますが、同じ関数を記述する方法が別にもあります。T は \texttt{AsJson} 特性を実装する必要があるジェネリック型であるということを明示的に指定します。


\begin{lstlisting}[numbers=none]
fn send_data_as_json<T: AsJson>(value: &T) { ... }
\end{lstlisting}

次に、使用する型を宣言し、それらの \texttt{AsJson} 特性を実装します。

\begin{lstlisting}[numbers=none]
struct Person {
    name: String,
    age: u8,
    favorite_fruit: String,
}

struct Dog {
    name: String,
    color: String,
    likes_petting: bool,
}

impl AsJson for Person {
    fn as_json(&self) -> String {
	    format!(
	        r#"{{ "type": "person", "name": "{}",
                  "age": {}, "favoriteFruit": "{}" }}"#,
	        self.name, self.age, self.favorite_fruit
	    )
    }
}

impl AsJson for Dog {
    fn as_json(&self) -> String {
	    format!(
	        r#"{{ "type": "dog", "name": "{}",
                  "color": "{}", "likesPetting": {} }}"#,
	        self.name, self.color, self.likes_petting
	    )
    }
}
\end{lstlisting}

これで \texttt{Person} と \texttt{Dog} の両方で \texttt{AsJson} 特性が実装されたので、それらを \texttt{send\_data\_as\_json} 関数の入力パラメーターとして使用することができます。

\begin{lstlisting}[numbers=none]
fn main() {
    let laura = Person {
    	name: String::from("Laura"),
	    age: 31,
	    favorite_fruit: String::from("apples"),
    };

    let fido = Dog {
	    name: String::from("Fido"),
	    color: String::from("Black"),
	    likes_petting: true,
    };

    send_data_as_json(&laura);
    send_data_as_json(&fido);
}
\end{lstlisting}

しかし、期待される特性を実装しない型を関数に渡すと、どうなるでしょうか? 新しい構造体を作成し、何が起こるかを見てみましょう。

\begin{lstlisting}[numbers=none]
struct Cat {
    name: String,
    sharp_claws: bool,
}

let kitty = Cat {
    name: String::from("Kitty"),
    sharp_claws: false,
};

send_data_as_json(&kitty);
\end{lstlisting}

コンパイラで次のエラーが発生します。


\begin{lstlisting}[numbers=none]
    error[E0277]: the trait bound `Cat: AsJson` is not satisfied
      --> src/main.rs:70:23
       |
    5  | fn send_data_as_json(value: &impl AsJson) {
       |                                   ------ required by this\\
                                        bound in `send_data_as_json`
    ...
    70 |     send_data_as_json(&kitty);
       |                       ^^^^^^ the trait `AsJson` is not\\
                                      implemented for `Cat`
\end{lstlisting}

このエラーが発生する原因は、 \texttt{send\_data\_as\_json} 関数において、 \texttt{AsJson} 特性を実装しない型を、その特性が期待されている場所で使おうとしたことにあります。

このユニットで使用したコードを確認するには、こちらの Rust Playground のリンクをご覧ください。

オプションの課題として、 \texttt{Cat} 型の \texttt{AsJson} 特性の実装を試みることができます。

 % 特性境界とジェネリック関数を使用する
\subsection{反復子を使用する}

ループを使用してコレクション型を反復処理する方法については既に学習しました。 今回は、Rust で反復自体の概念を処理する方法について、さらに掘り下げて確認します。

Rust では、すべての反復子で \texttt{Iterator} という名前の特性が実装されます。これは、標準ライブラリに定義されていて、範囲、配列、ベクター、ハッシュ マップなどのコレクションに反復子を実装する場合に使用されます。

この特性の核となる部分は次のようになります。

\begin{lstlisting}[numbers=none]
trait Iterator {
    type Item;
    fn next(&mut self) -> Option<Self::Item>;
}
\end{lstlisting}

\texttt{Iterator} には \texttt{next} というメソッドがあります。これを呼び出すと、 \texttt{Option<Item>} が返されます。 \texttt{next} メソッドからは、要素がある限り、 \texttt{Some(Item)} が返されます。 すべての処理が完了すると、反復が終了したことを示す \texttt{None} が返されます。

この定義には、この特性に関連付けられた型を定義する \texttt{type Item} および \texttt{Self::Item} という新しい構文が使用されていることに注目してください。 この定義は、 \texttt{Iterator} 特性のすべての実装には、関連付けられた \texttt{Item} 型の定義も必要であることを意味しています。これは \texttt{next} メソッドの戻り値の型として使用されます。 言い換えると、 \texttt{Item} 型は、 \texttt{for} ループ ブロック内の反復子から返される型になります。

\subsubsection{独自の反復子を実装する}

独自の反復子を作成するには、次の 2 つの手順が必要です。

\begin{enumerate}
\item 反復子の状態を保持する構造体を作成します。
\item その構造体に対して反復子を実装します。
\end{enumerate}

1 から任意の数 (\texttt{Counter} 構造体を作成するときに定義) までをカウントする \texttt{Counter} という名前の反復子を作成してみましょう。

最初に、反復子の状態を保持する構造体を作成します。 また、 \texttt{new} メソッドも実装して、開始する方法を制御します。


\begin{lstlisting}[numbers=none]
#[derive(Debug)]
struct Counter {
    length: usize,
    count: usize,
}

impl Counter {
    fn new(length: usize) -> Counter {
	    Counter {
	        count: 0,
	        length,
	    }
    }
}
\end{lstlisting}

次に、 \texttt{Counter} 構造体の \texttt{Iterator} 特性を実装します。 usize を使用してカウントすることになるので、関連付けられた \texttt{Item} 型をその型とすることを宣言します。

\texttt{next()} メソッドは、定義する必要がある唯一の必須メソッドです。 その本体内では、呼び出しのたびにカウントが 1 つずつインクリメントされます "(これがゼロから始めた理由です)"。 次に、カウントが終了したかどうかを確認します。 反復によって引き続き結果が生成されることを表するには、 \texttt{Option} 型の \texttt{Some(value)} バリアントを使用し、反復が停止される必要があることを表すには \texttt{None} バリアントを使用します。


\begin{lstlisting}[numbers=none]
impl Iterator for Counter {
    type Item = usize;

    fn next(&mut self) -> Option<Self::Item> {
    
        self.count += 1;
        if self.count <= self.length {
            Some(self.count)
        } else {
            None
        }
    }
}
\end{lstlisting}

\texttt{Counter} が機能することを確認するには、その \texttt{next} 関数を明示的に呼び出します。

\begin{lstlisting}[numbers=none]
fn main() {
    let mut counter = Counter::new(6);
    println!("Counter just created: {:#?}", counter);

    assert_eq!(counter.next(), Some(1));
    assert_eq!(counter.next(), Some(2));
    assert_eq!(counter.next(), Some(3));
    assert_eq!(counter.next(), Some(4));
    assert_eq!(counter.next(), Some(5));
    assert_eq!(counter.next(), Some(6));
    assert_eq!(counter.next(), None);
    assert_eq!(counter.next(), None);  // further calls to `next`
                                       // will return `None`
    assert_eq!(counter.next(), None);

    println!("Counter exhausted: {:#?}", counter);
}
\end{lstlisting}

しかし、この方法で \texttt{next} を呼び出すと、繰り返しになります。 Rust を使用すると、 \texttt{Iterator} 特性を実装する型に \texttt{for} ループを使用することができるので、次のようにしてみましょう。

\begin{lstlisting}[numbers=none]
fn main() {
    for number in Counter::new(10) {
        println!("{}", number);
    }
}
\end{lstlisting}

上記のスニペットを実行すると、コンソールに次の出力が出力されます。

\begin{lstlisting}[numbers=none]
    1
    2
    3
    4
    5
    6
    7
    8
    9
    10
\end{lstlisting}

\texttt{Iterator} 特性の完全な定義には他のメソッドも含まれていますが、それらは既定のメソッドです。 \texttt{next} を基にして作成されているので、無料で入手できます。

\begin{lstlisting}[numbers=none]
let sum_until_10: usize = Counter::new(10).sum();
assert_eq!(sum_until_10, 55);

let powers_of_2: Vec<usize> = 
     Counter::new(8).map(|n| 2usize.pow(n as u32)).collect();
assert_eq!(powers_of_2, vec![2, 4, 8, 16, 32, 64, 128, 256]);
\end{lstlisting}

このユニットの完全なコード例については、こちらの Rust Playground のリンクをご覧ください。
 % 反復子を使用する
\subsection{演習 - ジェネリック型を実装する}

この演習では、 \texttt{u32} 型の正の整数のみを受け入れる \texttt{Container} 構造体を、任意の型の値を保持できるジェネリック コンテナーに変換します。

\texttt{main} 関数内のコンテンツは編集しないでください。 この演習は、コードがコンパイルされると完了です。

\begin{lstlisting}[numbers=none]
struct Container {
    value: u32,
}

impl Container {
    pub fn new(value: u32) -> Self {
        Container { value }
    }
}

fn main() {
    assert_eq!(Container::new(42).value, 42);
    assert_eq!(Container::new(3.14).value, 3.14);
    assert_eq!(Container::new("Foo").value, "Foo");
    assert_eq!(Container::new(String::from("Bar")).value, String::from("Bar"));
    assert_eq!(Container::new(true).value, true);
    assert_eq!(Container::new(-12).value, -12);
    assert_eq!(Container::new(Some("text")).value, Some("text"));
}
\end{lstlisting}




 % 演習 - ジェネリック型を実装する
\subsection{演習 - 反復子を実装する}

この演習では、次のように、ベクターでグループ化されたシーケンス内の等しい項目を返す反復子を実装します。

\begin{itemize}
    \item 入力:\texttt{[ 1, 1, 2, 1, 3, 3 ]}
    \item 出力:\texttt{[ [1, 1], [2], [1], [3, 3] ]}
\end{itemize}

目標は、 \texttt{Group} 構造体に対する \texttt{Iterator} 特性 (反復の合間に \texttt{inner} フィールドにデータの状態を保持する役割を果たす) の実装の記述を完了することです。

この割り当てを実行するには値を比較する必要があるため、ジェネリック型 \texttt{T} を \texttt{PartialEq} 特性の実装元とする必要があります。 しかし、その部分は \texttt{impl<T: PartialEq>} セグメントで既に解決されているので、心配する必要はありません。

\begin{lstlisting}[numbers=none]
struct Groups<T> {
    inner: Vec<T>,
}

impl<T> Groups<T> {
    fn new(inner: Vec<T>) -> Self {
	    Groups { inner }
    }
}

impl<T: PartialEq> Iterator for Groups<T> {
    type Item = Vec<T>;

    // TODO: Write the rest of this implementation.
}

fn main() {
    let data = vec![4, 1, 1, 2, 1, 3, 3, -2, -2, -2, 5, 5];
    // groups:     |->|---->|->|->|--->|----------->|--->|
    assert_eq!(
	    Groups::new(data).into_iter().collect::<Vec<Vec<_>>>(),
	    vec![
	        vec![4],
    	    vec![1, 1],
	        vec![2],
    	    vec![1],
	        vec![3, 3],
	        vec![-2, -2, -2],
    	    vec![5, 5],
	    ]
    );

    let data2 = vec![1, 2, 2, 1, 1, 2, 2, 3, 4, 4, 3];
    // groups:      |->|---->|---->|----|->|----->|->|
    assert_eq!(
	    Groups::new(data2).into_iter().collect::<Vec<Vec<_>>>(),
	    vec![
	        vec![1],
    	    vec![2, 2],
	        vec![1, 1],
	        vec![2, 2],
    	    vec![3],
	        vec![4, 4],
	        vec![3],
	    ]
    )
}
\end{lstlisting}


 % 演習 - 反復子を実装する
\subsection{知識チェック}

次の質問に答えて、学習した内容を確認してください。

\begin{enumerate}
\item Rust の特性が役に立つのはどのような場合ですか?
\begin{itemize}
\item 省略可能なパラメーターを関数または構造体で受け取る必要がある場合。
\item 具体的な値ではなく、動作の観点から関数または構造体のパラメーターを指定する必要がある場合。
\item 借用チェッカーのコンパイル時間保証を避ける必要がある場合。
\item 値がその有効期間を過ぎても、引き続き有効にしておく必要がある場合。
\end{itemize}
\item 次の関数シグネチャはどういう意味ですか? \texttt{fn show\_on\_screen<T: Display>(data: T)}
\begin{itemize}
\item \texttt{data} パラメーターは任意の型にすることができます。
\item \texttt{data} パラメーターは、必要に応じて \texttt{Display} 特性を実装する任意の型とすることができます。
\item \texttt{data} パラメーターは、 \texttt{Display} 特性を実装する型のみに制限されています。
\item \texttt{data} パラメーターは省略可能です。
\end{itemize}
\end{enumerate}









 % 知識チェック
まとめ