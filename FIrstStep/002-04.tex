\subsection{Rust をインストールする}

Rust をインストールするために推奨される方法は、Rust ツールチェーン インストーラーの \texttt{rustup} を使用することです。 Web サイト rustup.rs にアクセスして、お使いのオペレーティング システムに合った手順を見つけてください。

\includegraphics[width=10cm]{rustup.eps}

Linux または macOS 上では、クリップボード アイコンを選択して curl コマンドをコピーします。 次に、コンピューターのターミナルまたはコマンド プロンプトを開いてコマンドを貼り付け、画面の指示に従います。 Windows 上では、インストーラーの指示に従います。

\begin{itembox}[l]{重要}
Rust には、Visual Studio 2013 以降の Microsoft C++ ビルド ツールが必要です。 Rust をインストールするには、事前にビルド ツールがインストールされている必要があります。 ビルド ツールをインストールする必要がある場合は、前のユニットの手順を参照してください。
\end{itembox}

Rust のラピッド リリース プロセスは 6 週間ごとに実行されます。非常に多くのプラットフォームがサポートされているので、任意の時点で利用できる Rust のビルドは多数存在します。 \texttt{rustup} を以前にインストールしたことがある場合は、\texttt{rustup update} コマンドを実行することで、Rust の最新の安定バージョンに更新できます。

\subsubsection{Rust のインストールを確認する}

Rust のインストールを完了したら、\texttt{rustc} コマンドと \texttt{cargo} コマンドを使用できるようになります。

\begin{itembox}[l]{注意}
次のコマンドはすべてのプラットフォームで動作します。
\end{itembox}

ターミナルまたはコマンド プロンプトで、次のコマンドを実行します。


\begin{lstlisting}[numbers=none]
rustc --version
\end{lstlisting}

この例のような出力になるはずです。

\begin{lstlisting}[numbers=none]
rustc 1.50.0 (cb75ad5db 2021-02-10)
\end{lstlisting}

次に、次のコマンドを実行します。

\begin{lstlisting}[numbers=none]
cargo --version
\end{lstlisting}

次のような出力が表示されます。

\begin{lstlisting}[numbers=none]
cargo 1.50.0 (f04e7fab7 2021-02-04)
\end{lstlisting}

いずれの出力行にも、利用可能な Rust と Cargo の最新の安定したバージョンに関する次の情報が含まれます。

\begin{itemize}
\item リリース番号
\item コミット ハッシュ
\item コミットの日付
\end{itemize}

この情報は次の形式で表示されます。

\texttt{<executable-name> <three-part-release-number> (<9-character-hash-code>\\ <4-digit-year>-<2-digit-month>-<2-digit-day>)}

この種類の出力が表示された場合、両方のインストールが成功しました。 この情報が表示されない場合、\texttt{PATH} 環境変数を確認してください。 \texttt{rustc.exe} と \texttt{cargo.exe} という実行可能ファイルが入っているフォルダーが必ず含まれるようにしてください。

\subsubsection{自分の知識をチェックする}

次の質問に答えて、学習した内容を確認してください。

\begin{enumerate}
\item Rust のインストールに使用するために推奨されているコマンドは何ですか?

\begin{itemize}
\item rinstall
\item rustup
\item rupdate
\end{itemize}

\item Rust ライブラリはどのくらいの頻度で更新されますか?
\begin{itemize}
\item 6 か月ごと
\item 3 か月ごと
\item 6 週間ごと
\end{itemize}
\end{enumerate}

